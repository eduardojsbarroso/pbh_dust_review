\documentclass[a4paper,11pt]{article}
\pdfoutput=1 % if your are submitting a pdflatex (i.e. if you have
             % images in pdf, png or jpg format)

\usepackage[T1]{fontenc} % if needed

%%%%%%%%%%%%%%%%%%%%%%%%%%%%%% My Packages %%%%%%%%%%%%%%%%%
\usepackage[english]{babel}
\usepackage{braket}
\usepackage{amsmath}
\usepackage{amsfonts}
\usepackage{amssymb}
\usepackage{physics,slashed}
\usepackage{graphicx}
\usepackage[toc,page]{appendix}
\usepackage{hyperref}
\usepackage[noabbrev]{cleveref}

\usepackage{xcolor}
\usepackage{mdframed}

\usepackage[hmarginratio=1:1]{geometry}

\begin{document}

We thank the referee for their remarks. Reviewing a long manuscript is a challenging
task, and we are grateful for the time and effort they dedicated to our work. In this
revised version, we have addressed all the points raised to improve the clarity and
quality of the manuscript, aiming to avoid any potential misunderstandings. All changes
in the manuscript are highlighted in red for ease of reference. Below is a summary of
the modifications made:
\begin{enumerate}
      \item In Sec.~2 and Sec.~3, we discussed how the background dynamics is affected
            by quantum effects and how the perturbations are calculated within the
            Bohmian framework. We have added a detailed explanation of how the Hubble
            function and its time derivative are computed in a Bohmian trajectory, as
            well as the role of energy density in calculating perturbations. This
            ensures clarity on how terms such as $1/(\bar{\rho} + \bar{p})$ remain
            well-defined when $\bar{H}^2$ or $\dot{\bar{H}}$ pass through zero.
      \item The revision mentions that the density contrast used is the same as in
            previous sections, but applies the approximation $w \ll 1$ for cold dark
            matter.
      \item In Sec.~4, we have added a detailed explanation of the assumptions made
            regarding the quantum-to-classical transition. We have clarified that the
            perturbations are treated as classical at the current stage of the work. If
            quantum effects were considered, a quantum Misner-Sharp model would be
            required, which is a complex problem not yet fully addressed in the
            literature.
\end{enumerate}



\end{document}
