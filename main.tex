\documentclass[a4paper,11pt]{article}
\pdfoutput=1 % if your are submitting a pdflatex (i.e. if you have
             % images in pdf, png or jpg format)

\usepackage{jcappub} % for details on the use of the package, please
                     % see the JCAP-author-manual

\usepackage[T1]{fontenc} % if needed


\title{\boldmath Primordial Black Hole Formation in a Dust Bouncing Model}

%%%%%%%%%%%%%%%%%%%%%%%%%%%%%% My Packages %%%%%%%%%%%%%%%%%
\usepackage[english]{babel}
\usepackage{braket}
\usepackage{amsmath}
\usepackage{amsfonts}
\usepackage{amssymb}
\usepackage{physics,slashed}
\usepackage{graphicx}
\usepackage[toc,page]{appendix}
\usepackage{hyperref}
\usepackage[noabbrev]{cleveref}

\usepackage{xcolor}
\usepackage{mdframed}

\usepackage[hmarginratio=1:1]{geometry}

%%%%%%%%%%%%%%%%%%%%%%%%%%%%%% My Comands %%%%%%%%%%%%%%%%%
\newcommand{\DifRel}{ \Delta^{\! \mathrm{r}} }
\newcommand{\DifAbs}{ \Delta^{\! \mathrm{a}} }
\newcommand{\SubMax}{ _{\mathrm{max} } }
\newcommand{\VS}{\mathbb{V}}
\newcommand{\condarg}[2]{\mleft(#1\middle\vert#2\mright)}
\renewcommand{\vec}{\mathbf}
\newcommand{\ncopt}{\texttt}
\newcommand\crefrangeconjunction{\textendash}
\renewcommand{\mathsterling}{\textrm{\textsterling}}
\newcommand{\dpar}[1]{\left(#1 \right)} 
\newcommand{\dcol}[1]{\left[#1 \right]} 
\newcommand{\dcha}[1]{\left\{#1 \right\}} 

%%%%%%%%%%%%%%%%%%%%%%%%%%%%%%%%%%%%%%%%%%%%%%%%%%%%%%%%%% 

\author[a,1]{E.J. Barroso,}\note{Corresponding author.}
\author[a]{L.F. Demétrio,}
\author[a]{S.D.P. Vitenti,}
\author[b]{Xuan Ye}

\affiliation[a]{Physics Department, Universidade Estadual de Londrina, \\Campus Universitário, CEP 86057-970, Londrina, Brasil,}
\affiliation[b]{ Department of Astronomy, Key Laboratory for Researches in Galaxies and Cosmology,
   School of Astronomy and Space Sciences, University of Science and Technology of China,
   96 JinZhai Road, Hefei, Anhui, 230026, China}

\emailAdd{eduardo.jsbarroso@uel.br}
\emailAdd{demetrio.luizfelipe@uel.br}
\emailAdd{vitenti@uel.br}
\emailAdd{yyyyy@ustc.edu.cn}

%%%%%%%%%%%%%%%%  Abstract    %%%%%%%%%%%%%%%%%%%%%%%%%%%
\abstract{Linear scalar cosmological perturbations have increasing spectra in the
	contracting phase of bouncing models. We study the conditions for which these
	perturbations may collapse into primordial black holes and the hypothesis that these
	objects constitute a fraction of dark matter. We compute the critical density
	contrast that describes the collapse of matter perturbations in the flat-dust bounce
	model with a parametric solution, obtained from the Lemaitre-Tolman-Bondi metric
	that represents the spherical collapse. We discuss the inability of the Newtonian
	gauge to describe perturbations in contracting models as the perturbative hypothesis
	does not hold in such cases. We carry the calculations for a different Gauge choice
	and compute the perturbations' power spectra numerically. Finally, assuming a
	Gaussian distribution, we compute the primordial black hole abundance with the
	Press-Schechter formalism and compare it with observational constraints. From our
	analysis, we conclude that the primordial black hole formation in a dust-dominated
	contracting phase does not lead to a significant mass fraction of primordial black
	holes in dark matter today.}
%%%%%%%%%%%%%%%%%%%%%%%%%%%%%%%%%%%%%%%%%%%%%%%%%%%%%%%%%%%%%%%%%%%%%


\begin{document}
\maketitle
\flushbottom

%%%%%%%%%%%%%%%%  SECTIONS  %%%%%%%%%%%%%%%%%%%%%%%%%%%
\section{Introduction}
\label{sec:intro}

Primordial Black Holes (PBHs) are believed to have been formed in the early universe
through the collapse of density fluctuations~\cite{Zel1967, Hawking1971, Hawking1974,
	Carr1974}. Due to their formation mechanisms, in the inflationary scenario, PBHs can
have a wider range of masses from around $M \sim 5\times 10^{-29} M_\odot$ if formed at
Planck time $t = 10^{-43}s $ and $M \sim 10^5 M_\odot$ if formed at $t\sim 1s$,
resulting in several different effects. The study of primordial black holes has yielded
insights into various phenomena. They could have contributed to the cosmological density
parameter~\cite{Carr1975, Hee1996}. Their influence in the Cosmic Microwave Background
Radiation (CMB) was studied in Ref.~\cite{Ricotti2008} and some recent works explored
the connection between evaporating PBHs and gravitational waves~\cite{Dom2021}. There
also have been discussions about whether some measurements of gravitational waves could
be attributed to primordial black holes~\cite{Wang2022}. Undoubtedly, the possibility
that primordial black holes constitute a significant fraction of cold dark matter
remains a focal point in current research~\cite{1975Natur}, since they are labeled as
non-baryonic as their formation takes place before the Big Bang Nucleosynthesis
(BBN)~\cite{Cyburt2003}. Recent observations of merging binary black holes with
unexpected mass ranges by the LIGO/Virgo collaborations indicate that they could have
originated from PBHs~\cite{Abbott2016, Abbott2019}.

Various mechanisms can result in primordial black hole formation (see
Ref.~\cite{Escriva2023} for a review). They can originate from the collapse of large
isocurvature perturbations of cold dark matter~\cite{Passaglia2022}, first-order phase
transitions~\cite{Khlopov1998, Liu2022}, critical collapse of matter
perturbations~\cite{Hawking1971, Carr1974}, and others. Based on the study of critical
phenomena and simulations, the formation mass of black holes in this context will
heavily depend on the cosmological model and the shape of perturbations
~\cite{Niemeyer1998}, which will directly impact the critical threshold $\delta_c$.
Thus, the density contrast and its critical value must be carefully studied.

In the inflationary scenario~\cite{Starobinskii1979, Guth1981, Bardeen1983, Linde1982},
several studies have analyzed the formation of PBH at the end of
inflation~\cite{Bullock1997, Yokoyama1998, Josan2010, Ballesteros2018, Wang2024} and in
the reheating phase~\cite{Carr2018, Martin2020} either as a probe for the inflationary
model or to analyze the abundance of PBH in dark matter. When confronting the results
with combined probes from observational data, it was found that the only acceptable PBH
mass range that allows these objects to entirely constitute dark matter is
$10^{-16}M_\odot - 10^{-12}M_\odot$~\cite{Carr2022, Villanueva2021}. Nonetheless, other
mass ranges that lead to a smaller fraction of dark matter have intriguing impacts to be
considered. For instance, in the mass range $10^6M_\odot$ to $10^{10} M_\odot$ where
PBHs only represent $0.1\%$ of DM, they could play a role in generating supermassive
black holes. In \cite{Garcia2017} it was also concluded that the possibility of dark
matter being constituted by supermassive primordial black holes is viable. However,
these results led researchers to study PBH formation in other cosmological models to
check the dependency of the mass constraints on the chosen models.

As is well known, the inflationary paradigm alleviates the initial condition problems of
the $\Lambda$-CDM model but does not completely solve
them~\cite{Guth1981,nelson2021bouncing, PatrickReview2}. For instance, one still needs
to assume that a small patch of spacetime was of FLRW type and evolved to become the
universe that we now observe, which is modeled via perturbation
theory~\cite{covariant_bardeen}. Alternative models to inflation have been considered,
mainly bouncing models (see~\cite{Gasperini1993, Gasperini1994, Lyth, Finelli,
	Wands1999, Brandenberger2001, Peter2002, Hwang2002, Vitenti2012, Vitenti2013,
	PatrickReview1} for extensive reviews on bouncing cosmology), which focus on solving the
singularity problem by introducing a contracting phase connected to the present
expanding phase trough a minimal scale factor: a bounce. Classical bouncing models,
where exotic matter or modifications of GR are considered, have been analyzed
extensively in the literature, mainly in Refs.~\cite{PatrickReview1, PatrickReview2}. In
these works, one sees that classical bounces are non-trivial to implement and lead to
undesired features such as instabilities and new singularities, which are associated
with the violation of the null energy condition.

One may also consider quantum bounces, where the quantization of gravity itself
eliminates the primordial singularity. In particular, since quantum bounces do not make
use of exotic matter such as the inflaton, there is no need for a reheating phase. This
has been achieved in previous works in the framework of Canonical Quantum
Gravity~\cite{nelson_peter_bouncing_original}. Other relevant proposals are Loop Quantum
Cosmology \cite{loop_quantum_gravity_perturbations_application,loop_phenomenology} and
String Gas Cosmology \cite{PatrickReview2, PatrickReview1}.

Some recent studies have analyzed the formation of PBH in bouncing
models~\cite{Carr2011, Corman2022, Chen2017, Chen2023, Quintin2016, Banerjee2022,
	Papanikolaou2024}. In this context, it is believed that the long duration of the
contracting phase plus the diminishing scale factor may lead to an enhancement of PBH
formation, resulting in a more significant contribution to the DM density. In
\cite{Quintin2016} it is concluded that a dust-dominated bouncing universe ($w > 1/3$,
being $w$ the fluid's equation of state) is robust against the formation of such
primordial black holes. In contrast, for a matter-dominated universe ($w \ll 1$), their
formation becomes relevant. A similar conclusion is achieved in \cite{Chen2023}, where
they got an enhanced production of PBHs near the bouncing point and utilized the
abundance of PBHs in DM to constrain the bouncing model. However, in the same work, it
is stated that these constraints are still not well understood due to a lack of
precision in numerical computations.

Furthermore, it is still not quite clear how to properly define the critical threshold
for PBHs in bouncing cosmology. Different from the inflationary scenario, the
contracting phase dynamics lead to growing perturbations that will collapse before
becoming super-Hubble. To circumvent this problem, in~\cite{Quintin2016} it is used the
argument that the black holes must be larger or equal to the Schwarzschild radius for
the given formation mass seeded by the perturbations. However, there is still the need
for a more precise definition of the critical contrast in the bouncing scenario.
Furthermore, the gauge definition used in previous works may not be the best choice as
it leads to large increasing spectra and thus a miscalculation of the energy-density
perturbations~\cite{vitenti2012large}.

In this work, we study the formation of PBH in a quantum dust non-singular bounce and
the hypothesis that these structures constitute a fraction of DM today. We consider the
single barotropic fluid quantum bouncing model developed in
\cite{nelson_peter_bouncing_original, nelson2000bohm, nelson2021bouncing} using
Canonical Quantum Gravity, which is a conservative approach to the quantization of
General Relativity and should hold as an effective theory up to a certain energy scale
\cite{nelson2021bouncing}. We shall focus on the critical collapse of matter
perturbations characterized by the density contrast $\delta$, such that the
perturbations collapse to form black holes when they achieve a given threshold ($\delta
	> \delta_c$). To perform this analysis, we need to compute the spectra of the
perturbations and the critical threshold needed for the distribution of PBHs. We compute
the critical threshold in a more detailed approach, using the Tolman-Bondi-Lemaitre
metric~\cite{Tolman1934, Lemaitre1933} in a similar way as done in~\cite{Martin2020}.
The perturbations' power spectra will be obtained through an algorithm that computes the
valid interval for the adiabatic approximation and solves the dynamics of the
perturbations. With these results, we will compute the abundance of PBHs and compare our
results with other works in bouncing cosmologies.

This paper is divided as follows: Section~\ref{sec:bounce} is devoted to reviewing the
quantum bouncing model. In Sec.~\ref{linearpert} we define the quantized perturbations
around our background and discuss the Gauge problem. In Sec.~\ref{sec:formation} we
describe the general formation criteria for PBHs formed through critical collapse and
compute the critical threshold for the the bouncing scenario. In the same section, we
compute the PBH mass fraction and abundance in the bouncing model. We discuss and
conclude our results in Sec.~\ref{sec:discussion}.

%%%%%%%%%%%%%%%%%%%%%%%%%%%%%%%%%%%%%%%%%%%%%%%%%%%%%%%%%%

\section{Flat-Dust Quantum Bouncing Background}
\label{sec:bounce}

In this section, we briefly discuss our adopted quantum bouncing background model, for
which we follow mostly Refs.~\cite{nelson_peter_bouncing_original, fluidgeral,
	nelson2021bouncing}. The primary motivation behind considering quantum bounces lies in
the necessity to evade the primordial singularity at the end of the contracting phase.
Classical General Relativity, as described by the Penrose-Hawking singularity theorems,
requires the violation of the null energy condition $\rho + p \geq 0$ to avoid the
singularity, a condition that often leads to instabilities in classical bouncing models.

Our quantum bouncing model is established through the canonical quantization of General
Relativity. The flat-dust bounce is thus a contracting universe model with a negative
effective energy density that dominates near the bounce, while classical behavior
prevails on larger scales. Nonetheless, the quantum contribution is sufficient to avoid
the singularity. This approach represents a conservative method in addressing quantum
gravity, as it applies the usual Dirac quantization techniques of constrained systems to
General Relativity \cite{nelsonhamiltonian}.

To apply canonical quantization to GR, we adopt the standard Hamiltonian formalism.  The
complete Wheeler-De Witt (WdW) equation that will arise from the Dirac quantization
poses several challenges that can be solved by assuming some additional hypothesis
~\cite{nelson2000bohm,halliwell1990introductory,nelson2021bouncing,nelson_bohm2023},
which we now discuss.  However, we should emphasize that the actual bouncing mechanism
is irrelevant to our results since the PBHs with relevant scales are formed way before
the quantum phase.

\subsection{Additional Hypothesis for Quantization}

The first problem arises because the WdW equation is formulated on superspace, which
represents the space of all possible metrics modulo diffeomorphisms and remains poorly
understood~\cite{halliwell1990introductory,dewitt1967}. Therefore, we shall only perform
quantization on a well-behaved sub-space that possesses the required symmetries, a
procedure that is known as a mini-superspace quantization. Thus, we quantize only the
sub-space of all possible flat FLRW geometries whose line element is given by the form
\begin{align}
	\label{physmetric}
	ds^2 = -N^{2}dt^2 + \bar{a}^2 \delta_{ij}dx^i dx^j
	,\end{align}
where $\bar{a}$ is the scale factor, $N$ is the lapse function and $\delta_{ij}$ is the
Kronecker delta. In this case, all information about the metric is stored in only one
degree of freedom, the scale factor $\bar{a}(t)$~\cite{nelson2021bouncing}.

Another problem is the lack of an explicit time evolution in the Hamiltonian of the
theory. This can be seen by attempting to write the conventional Schrödinger equation
for the total Hamiltonian and stating the lack of a clear time evolution parameter. This
fact is commonly referred to as the Problem of Time in Quantum Gravity ~\cite
{patrick_time_review,nelson_peter_bouncing_original,bianchi_time}. To define a
non-trivial propagator, we shall use an intrinsic variable of the system whose classical
evolution is monotonic. Subsequently, we require that the classical concept of time
emerges from this variable in the classical limit. Let us now apply such considerations
and discuss the solutions for this quantization procedure.

\subsection{Wheeler-DeWitt Equation Solutions}

{\color{red} We start by applying the aforementioned considerations to a flat,
	homogeneous, and isotropic universe containing a single perfect fluid characterized by
	its pressure $\bar{p}$ and energy density $\bar{\rho}$, along with the equation of state
	$\bar{p} = w\bar{\rho}$, with a constant $w$. Subsequently, we specialize in the dust
	case where $w \approx 0$. In practice, we consider a barotropic fluid with a constant
	equation of state, following the Schutz formalism as described in
	Ref.~\cite{fluidgeral}. In this section we will briefly review the quantization of the
	flat-dust bouncing model, focusing on the background dynamics.

	To proceed with the quantization, we first derive the system's Hamiltonian
	from its corresponding Lagrangian. The Lagrangian for the system is given by
	\begin{align}\label{lagrangian}
		L & = \int \dd^3x \sqrt{-\bar{g}} \left[\frac{\bar{R}}{2\kappa}
			+ \bar{p}(\bar{v}, \bar{s})\right],
	\end{align}
	where, $\bar{g}$ denotes the determinant of the metric, $\bar{R}$ is the background
	Ricci scalar, and $\kappa = \frac{8 \pi G}{c^4}$, where $c$ is the speed of
	light and $G$ is the gravitational constant. We set $c = 1$ for simplicity. The
	term $\bar{p}$ represents the fluid pressure, which is expressed as a function of
	the specific enthalpy $\bar{v}$ and specific entropy $\bar{s}$.

	Using the Schutz formalism, the enthalpy can be decomposed into scalar potentials:
	\begin{align}
		\bar{v}_\mu & = \bar\nabla_\mu \varphi_1
		+ \varphi_2 \bar\nabla_\mu \varphi_3
		+ \varphi_4 \bar\nabla_\mu \bar{s},
		\quad \bar{v} \equiv \sqrt{-\bar{v}_\mu \bar{v}^\mu},
	\end{align}
	where $\varphi_i$ are four independent scalar potentials. Varying the Lagrangian with
	respect to these potentials and the specific entropy yields all thermodynamic relations,
	including the perfect fluid energy-momentum tensor, entropy conservation, and particle
	number conservation.

	Since we assume a barotropic fluid with a constant equation of state $\bar{p} = w
		\bar{\rho}$, it follows that $\partial \bar{p} / \partial \bar{s} \vert_{\bar{\rho}} =
		0$, where $\bar{\rho}$ is the energy density. This assumption simplifies the pressure to
	the form $\bar{p} = f(\bar{s}) \bar{v}^{(1+w)/w}$.

	After performing a Legendre transformation and solving the constraints, the Lagrangian
	simplifies to:
	\begin{align}
		L          & = \Pi_a \dot{\bar{a}} + \Pi_{\varphi_1} \dot{\varphi}_1 +
		N \left( \frac{\kappa \Pi_a^2}{12 V \bar{a}} - V \bar{a}^3 \bar{\rho} \right), \\
		\Pi_a      & = -\frac{6 V \bar{a} \dot{\bar{a}}}{N \kappa}, \qquad
		\Pi_{\varphi_1} = V \bar{a}^3 f \frac{1+w}{w}\bar{v}^{1/w},                    \\
		\bar{\rho} & = \frac{f}{w\bar{a}^{3(1+w)}} \left( \frac{w}{1+w}
		\frac{1}{Vf} \right)^{1+w}\Pi_{\varphi_1}^{1+w},
	\end{align}
	where $V$ is the conformal volume. Since the fluid term has the form
	$\Pi_{\varphi_1}^{1+w}$ and the momentum term for the scale factor includes a factor
	dependent on $\bar{a}$, we introduce a final canonical transformation to define new
	momenta, $\Pi_q$ and $\Pi_T$
	\begin{align}
		\Pi_q & = \Pi_{\bar{a}} \bar{a}^{(-1+3w)/2},                                                 \\
		\Pi_T & = \frac{V f}{w} \left( \frac{w}{1+w} \frac{1}{Vf}\right)^{1+w} \Pi_{\varphi_1}^{1+w}
	\end{align}.

	In the above framework, the total Hamiltonian takes the form
	\begin{align}
		\label{total_hamiltonian_FLRW}
		{\cal H} & = \frac{N}{\bar{a}^{3w}}\dpar{\Pi_{T}- \frac{\kappa\Pi_q^{2} }{12}} ,
	\end{align}
	Canonical quantization is then performed by promoting classical variables to operators
	satisfying the canonical commutation relations. This process yields the following
	Wheeler-DeWitt equation for the wave-function of the universe  $\Psi(\bar{a}, T)$
	(see Ref.~\cite{nelson_peter_bouncing_original}):
	\begin{equation}
		\label{wdweq}
		i\hbar\frac{\partial}{\partial T}\Psi(q,T) -
		\frac{\kappa\hbar^2}{12}\frac{\partial^{2}}{\partial q^{2}}\Psi(q,T)  = 0 ,
	\end{equation}
	such that a specific operator factor ordering was chosen to preserve the symmetries of
	the classical system \cite{halliwell1990introductory}. Also, note that our last
	canonical transformation leads to the following variable
	\begin{equation}
		q = \frac{2\bar{a}^{\frac{3}{2}\dpar{1-w} } }{ 3\dpar{1-w } }.
	\end{equation}
	In the WdW equation above, we circumvent the problem of time by interpreting the
	parameter $T$ as an intrinsic time variable that is monotonically related to the
	classical cosmic time in the classical limit. This allows us to interpret the equation
	as a Schrödinger-like equation.
}

Equation~\eqref{wdweq} resembles a time-reversed free particle Schrödinger equation and,
with appropriate boundary conditions, its solutions are wave functions of the scale
factor $\bar{a}$. We turn to the De Broglie-Bohm
interpretation~\cite{mukhanov2005physical, nelson2000bohm}, such that assuming a Gaussian
wave-function $\Psi(q, T)$, the Bohmian trajectory solution translates to the scale
factor reads
\begin{equation}
	\label{bohm_scale_factor}
	\bar{a}(T) = \bar{a}_{B}\dcol{1 + \dpar{ \frac{T}{T_{ B}} }^2 }^{{\frac{1}{3}\frac{1}{\dpar{1-w}} } }  ,
\end{equation}
where $\bar{a}_{B}$ is an integration constant that represents the minimum scale factor
value and $T_B$ is a small arbitrary constant related to the time scale of the bounce.
	{\color{red} To simplify the analysis, we choose the lapse function $N = \bar{a}^{3w}$,
		but this choice is only possible after solving the Bohmian trajectory. The reason is
		that, initially, we cannot select $N$ as an operator function of the scale factor,
		However, once we have the Bohmian trajectory $\bar{a}(T)$, we can make this choice,
		using it purely for convenience in the analysis. In our model, $T$ extends from $T =
			-\infty$ in the far past to $T = 0$ at the bounce, where the scale factor $\bar{a}$
		reaches its minimum. } Remarkably, $\bar{a}(T) \neq 0$ for all $T$, which means that
this model is non-singular and represents an eternal
universe~\cite{nelson_peter_bouncing_original}. For a detailed derivation, see
\cite{nelson2021bouncing}.

{\color{red} It is important to clarify that in Ref.~\cite{nelson2021bouncing}, the
author selects a Gaussian wave function for the scale factor at the bounce. However,
this does not imply that this choice is the initial condition for our model. Generally,
a Gaussian wave packet of a one-dimensional free particle will develop a time-dependent
phase proportional to $q^2$. Thus, it is always possible to begin with a Gaussian wave
function at any time and define the bounce as the moment when this phase factor is zero.
For instance, in Ref.~\cite{Peter2016a}, the authors choose a Gaussian wave function at
any time and fix the phase using $H_0$. }

Given that we focus solely on a contracting universe model filled with dust, we set $w
	\approx 0$, simplifying the time variable $T$ to the conventional cosmic time $t$. With
the scale factor obtained in Eq.~\eqref{bohm_scale_factor}, we derive its associated
Hubble function
\begin{equation}\label{bohm_hubble}
	\bar{H}(t) \equiv \frac{1}{\bar{a}}\frac{d\bar{a}}{dt} = \frac{2}{3}\frac{t}{ \dpar{ t^{2} + t^{2}_{b} } }\, ,
\end{equation}
and invert \eqref{bohm_scale_factor} to obtain
\begin{equation}\label{bohm_time}
	t(\bar{a}) = \pm t_{b}\sqrt{ \dpar{\frac{\bar{a}}{\bar{a}_{b}}}^{3} - 1 } .
\end{equation}

Eliminating the time $t$ using \eqref{bohm_time} in \eqref{bohm_hubble}, we find
\begin{equation}
	\label{friedman_scale}
	\bar{H}^{2} = \frac{ 4 }{ 9t^{2}_{b}}\dpar
	{ \frac{ \bar{a}^{3}_{b} }{ \bar{a}^{3} } - \frac{\bar{a}_{b}^{6}}{ \bar{a}^{6} } }\, ,
\end{equation}
which is equivalent to the Friedmann equations
\begin{equation}\label{effective_friedmann}
	\bar{H}^{2} = \frac{\kappa}{ 3 }\bar\rho - \bar{H}^{2}_{0}\Omega_{q0}\bar{a}^{-6},
\end{equation}
where $\bar\rho$ is the dust energy density. {\color{red} Also, from the derivative of
		Eq.~\eqref{friedman_scale},
		\begin{align}
			\dot{\bar{H}} = - \frac{\kappa}{2} \bar\rho + 3\bar{H}^{2}_{0}\Omega_{q0}\bar{a}^{-6}.
		\end{align}
	}

In the above equations, {\color{red} the overdot represents the time derivative and} there is an additional term
$-\bar{H}^{2}_{0}\Omega_{q0}\bar{a}^{-6}$ when compared to the usual Friedmann equation
for a classical universe. {\color{red} In our model, the scale factor dynamics resemble
		a typical Friedmann equation with a total energy density $\rho_{T} = \bar{\rho} +
			\rho_{q},$ where $\bar{\rho}$ is the dust energy density, and $\rho_{q} = -\Omega
			\rho_{c} \bar{a}^{-6}$ denotes an effective negative energy density with an equation of
		state $w_q = 1$ that accounts for quantum effects to avoid the primordial
		singularity~\cite{vitenti2012large}. It is important to note that $\rho_{q}$ is not a
		physical energy density but rather an effective one that emerges from the Bohmian trajectory. In
		the perturbative analysis, the energy density will always be associated only with
		$\bar{\rho}$. For large-scale factors $\bar{a} \gg \bar{a}_{b} $,
		Eq.~\eqref{effective_friedmann} and Eq.~\eqref{bohm_scale_factor} reduce to the
		classical dust dominated a flat FLRW universe. With this, we conclude the analysis of our
		background quantum bouncing model and move to its associated perturbations. In the next
		sections, we will consider the above quantum bouncing model with $w\approx 10^{-10}$ and
		$\bar{a}(t_0)/a_b = 10^{35}$ where $t_0$ is the time during the contracting phase where
		$H(t_0) = -H_0$ for $H_0 = 70 \, \text{km/s/Mpc}$. Moreover, when solving the equations
		numerically, we use the exact expressions in terms of $w$ without approximation.}


%%%%%%%%%%%%%%%%%%%%%%%%%%%%%%%%%%%%%%%%%%%%%%%%%%%%%%%%%%

\section{Linear Scalar Perturbations}
\label{linearpert}
While the universe appears homogeneous and isotropic on large scales, the FLRW metric falls short of providing a precise description of our universe, which presents inhomogeneities that are associated with structure formation, e.g. galaxy clusters, black holes, stars, and others. To characterize the physical universe, we study linear scalar perturbations around our flat contracting background metric described in the previous section. We will then investigate how said perturbations may seed the formation of primordial black holes in the subsequent section. We will from now on characterize background quantities with an overbar and perturbed (physical) quantities without it.

\subsection{Gauge Invariant Perturbations}

We follow mainly the perturbation theory developed in Refs.~\cite{covariant_bardeen,
	Vitenti2012}. Assuming only scalar perturbations, the total metric of the physical
space-time is given by
\begin{align}
	\label{physmetric-pert}
	\dd s^2 = -(1- 2 \phi)\dd t^2 + \bar{a}\bar{D}_i \mathcal{B} \dd t \dd x^i +
	\bar{a}^2(1-2 \psi)\delta_{ij}\dd x^i \dd x^j - \bar{D}_i\bar{D}_j \mathcal{E} \dd x^i \dd x^j
	.\end{align}
Here, $\phi, \psi, \mathcal{B}$ and $\mathcal{E}$ denote the scalar metric perturbations
that we will assume to be much smaller than one \cite{vitenti2012large}. Also,
$\bar{D}_i$ is the spatial covariant derivative in the $i$-th direction. Note that our
barotropic fluid has no anisotropic pressure.

	{\color{red} In the perturbative treatment, we continue to use the Schutz formalism
		and construct the Lagrangian as in Eq.~\eqref{lagrangian}, but now incorporating the
		perturbed metric. The perturbed Lagrangian and associated variables are defined as
		in Ref.~\cite{Vitenti2012}. Note that in the perturbed Lagrangian, the perturbations
		are always linked to the physical quantities and their corresponding background
		values. The effective energy density and pressure do not fluctuate; instead, the
		quantum effects from the background are incorporated through the use of the
		background Bohmian trajectories.}

The metric perturbations are gauge-dependent variables. In cosmology, a gauge can be
seen as the freedom in how we connect, or map, the background and physical manifold, and
how we choose our coordinate system~\cite{vitenti2012large}. Since GR is a covariant
theory, this freedom may then be interpreted as a gauge, which may lead to an ambiguous
description of perturbations. Depending on the foliation that characterizes the manifold
hyper-surfaces and how we define the perturbations around our background, the physical
quantities may have different values \cite{covariant_bardeen}. Thus, it is recommended
that we work with gauge independent variables to carry out our computations and go back
to the physical variables at the end when necessary \cite{mukhanov2005physical}.

We define gauge invariant quantities by analyzing their transformations under gauge transformations, such that we can combine different gauge-dependent quantities to form new invariant ones~\cite{covariant_bardeen}. However, this leads to a freedom on the variable definitions since many combinations of variables may lead to gauge invariant quantities. With this in mind, we define the Bardeen gauge invariant variables~\cite{Bardeen1980}
\begin{align}
	\label{bardeen1}
	\Phi & \equiv \phi + \dot{\delta \sigma}\, , \\
	\label{bardeen2}
	\Psi & \equiv \psi - \Bar{H}\delta \sigma
	,\end{align}
with
\begin{align}
	\delta\sigma = -(\dot{\mathcal{E}} - \mathcal{B}) + {2 \Bar{H}\mathcal{E}}
	.\end{align}
It is important to notice that the new variables in
Eqs.~\eqref{bardeen1}-\eqref{bardeen2}, and other gauge invariant variables do not have
a physical meaning unless a gauge is chosen. For instance, in the case of the Newtonian
gauge, $\delta\sigma = 0$ and $\Phi$ represents the Newtonian potential. Hence we must
define our gauge invariant quantities such that they represent our desired physical
variables when we assume a particular gauge choice.

We are mostly interested in the energy density perturbations that collapse to form PBHs.
These perturbations can be examined using the density contrast, defined as
\begin{align}
	\label{densitycon}
	\delta & \equiv \frac{\delta \rho}{\bar{\rho} + \bar{p}}
	,\end{align}
where $\delta\rho$ is the perturbation to the background energy density $\bar\rho$. The
density contrast provides a normalized measurement of the energy density perturbation
around the background matter density field. However, once again we are interested in its
gauge-invariant form
\begin{align}
	\label{deltarhoinvariant}
	\tilde{\delta\rho} & \equiv \delta\rho - \mathcal{V} \dot{\Bar{\rho}}
	,\end{align}
where $\mathcal{V}$ is the velocity perturbation of the fluid, which has its gauge
invariant form
\begin{align}
	\tilde{\mathcal{V}} & \equiv \mathcal{V} + \delta\sigma
	.\end{align}

Using the background Einstein equations, we can relate the density contrast to the
Bardeen variables and the gauge invariant curvature perturbation $\zeta$ through the
following relations~\cite{Mukhanov1992, Vitenti2013}
\begin{align}
	\Psi                                                    & =\Phi,
	\\
	\label{deltarho}
	-\frac{2\bar{D}^2 \Phi}{3  \kappa (\bar{\rho}+\bar{p})} & =   \frac{{\tilde{\delta\rho}}} {3(\bar{\rho}+\bar{p})}
	,
	\\
	\label{vrelation}
	\zeta                                                   & \equiv \Psi  + \bar{H} \tilde{\mathcal{V}}
	\\
	\label{zeta2}
	\zeta                                                   & = \frac{3  \bar{a}^3}{N^2z^2 c_s^2 } \left[\frac{\partial}{\partial t}\left(\frac{\Phi }{3\bar{H}}\right) + \frac{\Phi}{3}\right]
	,                                                                                                                                                                                            \\
	\label{phiz}
	\bar{D}^{2} \Phi                                        & =  - \frac{z^2\bar{H}}{\bar{a}^3} \dot{\zeta} = -\frac{\bar{H}}{2\bar{a}^3} \Pi_\zeta
	.\end{align}
\textcolor{red}{In the above, $c^{2}_s=\dpar{ \frac{\partial\bar{\rho}}{\partial\bar{p}} }_{S}= w$ is the speed of sound, $\Pi_\zeta$ is the conjugated momenta related to $\zeta$ and
	\begin{align}
		\label{zdef}
		z^2=\frac{\kappa \bar{a}^3 (\bar{\rho} + \bar{p})}{2\bar{H}^2 c_{s}^2}
		.\end{align}}
For completeness, we also write the relation between the curvature perturbation and the
usual Mukhanov-Sasaki variable $v$, that is,
\begin{align}
	\label{msv}
	v & \equiv - \zeta z \sqrt{\frac{2}{\kappa}}
	.\end{align}
In the next section, we will establish a connection between PBH formation and density contrast. For now, it suffices to understand that the excess energy density associated with the perturbations leads to black hole formation, and we can measure such excess through the density contrast. Thus our definition in Eq.~\eqref{deltarhoinvariant} is extremely important and we will now analyze it.

Note from Eq.~\eqref{deltarhoinvariant} that if we choose a gauge where $\mathcal{V} =0$, $\Tilde{\delta\rho}$ becomes the physical density perturbation, which leads to an easier interpretation of this quantity. Also, the choice of $\mathcal{V} =0$ will be well suited to connect our perturbation theory with the Lemaitre-Toman-Bondi metric discussed in App.~\ref{appgauge}.  Other works have defined the gauge invariant density perturbation as
\begin{align}
	\label{newtoniandeltarho}
	\Tilde{\delta\rho}^{N} & \equiv \delta\rho - \delta\sigma \dot{\Bar{\rho}}
	,\end{align}
so that this quantity becomes the physical variable in the Newtonian Gauge, where $\delta\sigma = 0$ and $\Phi = \phi$. However, in Ref.~\cite{vitenti2012large}, it was shown that for bounce models with long contracting phases, regardless of the bounce type, the Bardeen perturbation $\Phi$ grows larger than one and invalidates the perturbative series, as $\phi$ also grows larger than one in the Newtonian Gauge. Additionally, the definition in Eq.~\eqref{newtoniandeltarho} leads to
\begin{align}
	\zeta & = \Phi + \frac{2\bar{D}^2 \Phi}{3  \kappa (\bar{\rho}+\bar{p})}+  \frac{{\tilde{\delta\rho}^N}} {3(\bar{\rho}+\bar{p})}
	,\end{align}
which implies that $\tilde{\delta\rho}^N$ grows with $\Phi$ and thus has larger spectra
as well as in this gauge. Hence, the Newtonian gauge is not a valid choice for bounce
models as it would lead to miss-calculation of the physical quantities whose values
would be inflated. We shall avoid this choice in this work and stick with the definition
in Eq.~\eqref{deltarhoinvariant}. We now compute the density contrast modes, which
require the perturbative equations of motion of the model.

\subsection{Classical Equations of Motion}

The perturbations are described by the Einstein-Hilbert action expanded up to second order, resulting in the Mukhanov-Sasaki Lagrangian
\begin{align}
	\label{lagms}
	L_{MS} & = \int \dd^3\textbf{x} \frac{1}{ \kappa} \left({\dot{\zeta}}^2 z^2 + c_s^2 z^2 \zeta\Delta \zeta\right)
	,\end{align}
where $\Delta = \bar{D}^2$ is the spatial Laplacian operator and $z$ is given by Eq.~\eqref{zdef}.
To compute the density contrast in Eq.~\eqref{deltarho}, we will need the curvature modes related to the Mukhanov-Sasaki Lagrangian. The extremization of Eq.~\eqref{lagms} in Fourier space yields the equation of motion
\begin{align}
	\label{four}
	\Ddot{\zeta}_{k}+ 2\frac{ \dot{z} }{ z } \dot{\zeta}_k + \frac{k^2}{\bar{a}^2} \zeta_k & =0
	,\end{align}
where $k$ is the comoving wave number of the modes. Since obtaining analytical solutions to \eqref{four} is non-trivial, we will employ a numerical code to compute the modes, which requires the use of dimensionless variables. In dimensionless units, denoted with a subscript $A$, we redefine our variables as\footnote{A derivative of a dimensionless variable will also be dimensionless. However, we use the same dot notation.}
\begin{align}
	\zeta_{k_A, A}   & \equiv \frac{\zeta_{k}}{\sqrt{ \kappa \hbar R_H}},           \\
	\Pi_{\zeta_k, A} & \equiv \frac{\Pi_{\zeta_k} \sqrt{R_H}}{\sqrt{\kappa \hbar}}, \\
	\label{kadm}
	k_A              & \equiv k R_H,                                                \\
	t_A              & \equiv \frac{t}{R_H}
	,\end{align}
where $l_p$ is the Planck length and $R_H = \frac{1}{\bar{H}_0}$. The equation of motion becomes
\begin{align}
	\label{eomosc}
	\dot{\Pi}_{k_A, A}+  \frac{2 c^2_s z^2 k_A^2}{\bar{a}^2  } \zeta_{k_A, A} & =0 \nonumber~~ or \\
	\dot{\Pi}_{k_A, A}+  2 \omega_{k_a}^2 z^2  \zeta_{k_A, A}                 & =0
	.\end{align}
Here, $\omega_A^2 \equiv \frac{c_s^2 k_a^2}{\bar{a}^2}$ is the dimensionless frequency and the dimensionless conjugated momenta are given by
\begin{align}
	\Pi_{k_A, A} & = 2 z^2 \dot{\zeta}_{k_A, A}
	.\end{align}
Eq.~\eqref{eomosc} can be interpreted as a harmonic oscillator with a time-dependent frequency $\omega_{k_a}(t_a)$ and mass $m(t_a) = 2z^2$. From here on, we are going to use the dimensionless variables with the same notation as described above.

Before solving this system numerically, we need to discuss the quantized version of our modes. This is because we need a prescription to set initial conditions of the perturbative variable $\zeta$. We do that by imposing initial vacuum conditions for the quantum fields, and thus, the problem turns into the problem of defining an appropriate vacuum state.

\subsection{Quantization}
%%%%%%%%%%%%%%%%%%%%%%%%

In this model, the background is characterized by a quantum contracting phase where an
effective quantum fluid dominates near the bounce. Consequently, we seek quantized
perturbations for consistency, although both theories can be viewed independently. In
particular, the use of quantized perturbations evolving on classical backgrounds has
been widely used since the results of Mukhanov and Chibisov \cite{mukhanov1981quantum}
and Hawking \cite{hawking1982quantum_fluctuations} to derive the power spectrum of
primordial perturbations, which are in turn used to describe the formation of structure
in our universe. {\color{red} In our model, we quantize both the background and the
		perturbations. Since we adopt the De Broglie-Bohm interpretation for the background,
		quantum effects are described in terms of the Bohmian scale factor, as shown in
		Eq.~\eqref{bohm_scale_factor}. This allows us to treat quantum perturbations evolving on
		this scale factor. To demonstrate the consistency of this approach,
		Ref.~\cite{Peter2005} derived the second-order Lagrangian for tensor perturbations
		without assuming classical equations of motion for the background. Furthermore, by using
		a factorized wave function for the perturbations and the background, the authors showed
		that these perturbations evolve conditioned on the Bohmian scale factor. In
		\cite{Vitenti2013}, the equivalent Lagrangian for scalar fluid perturbations was
		derived and is the one we use in this work.}

The use of quantum fields to describe primordial perturbations means that said fields
will have statistical properties. We may then partition the universe into local spatial
regions and consider each one as a realization of a random process and compare its
statistical properties with our theoretical predictions.\footnote{Here, an important
	remark must be made: in inflationary models, one usually postulates this quantum to
	classical statistical connection, but the specific mechanism that converts a quantum
	universe to a classical one is still an open problem
	\cite{nelson2021bouncing,mukhanov2005physical}. In our model, even if the perturbative
	level predictions do not depend strongly on the bouncing mechanism if one assumes the De
	Broglie-Bohm interpretation applied to Canonical Quantum Gravity, this problem is
	automatically solved \cite{nelson_bohm2023}. }

We proceed to quantize our fields by promoting them to Hermitian operators that act on a
Fock space. In terms of the usual Fourier mode expansion, our quantum operators, denoted
with a hat superscript~$\hat{ }$~from now on, are given by
\begin{align}
	\label{vexp}
	\hat{\zeta}(\textbf{x}, t) & = \frac{1}{(2\pi)^{\frac{3}{2}}}\int \mathrm{~d}^3{\textbf{k}} \left(e^{i{\textbf{k}}.\textbf{x}}\zeta_{k}^{*}(t) a_k + e^{-i{\textbf{k}}.\textbf{x}}\zeta_{k}(t) a^{\dagger}_k\right)
\end{align}
and
\begin{align}
	\label{pvexp}
	\hat{\Pi}_\zeta(\textbf{x}, t) & = \frac{1}{(2\pi)^{\frac{3}{2}}}\int \mathrm{~d}^3{\textbf{k}} \left(e^{i{\textbf{k}} .\textbf{x}}\Pi_{\zeta_k}^{*}(t) a_k + e^{-i{\textbf{k}} .\textbf{x}}\Pi_{\zeta_k}(t) a^{\dagger}_k\right)
	,\end{align}
where $\textbf{k}$ is the momentum vector with modulus $k$ and $a_k$ and $a_k^{\dagger}$ are the annihilation and creation operators respectively. Demanding that the quantum fields satisfy the canonical commutation relations
%
\begin{align}\label{commutation_relations}
	\left[\hat{\zeta}(\textbf{x}, t) , \hat{\Pi}_\zeta(\textbf{y}, t) \right] = i\hbar\delta(\textbf{x} - \textbf{y})\, ,
\end{align}
%
and that the complex modes satisfy
\begin{align}
	\label{basec}
	\dot{\zeta}_{k} \zeta_{k}^{*} - \zeta_{k}\dot{\zeta}_{k}^{*} & =i\hbar\, , \\
	%
	\dot{\Pi}_{k} \Pi_{k}^{*} - \Pi_{k}\dot{\Pi}_{k}^{*}         & =i\hbar\, ,
	.\end{align}
%
imply that the creation and annihilation operators $ a_k,a_k^{\dagger} $ satisfy the usual creation and annihilation algebra
\begin{align}
	\left[{a}_{k_1}, {a}_{k_2}\right] & = \left[{a}^\dagger_{k_1}, {a}^\dagger_{k_2}\right] = 0
\end{align}
and
\begin{align}
	\label{acom}
	\left[{a}_{k_1}, {a}^\dagger_{k_2}\right] = \delta(k_1 - k_2)
	,\end{align}
where $\delta(k_1 - k_2)$ is the dirac delta between the momenta.

Equation~\eqref{basec} represents the only constraint for choosing the basis of our problem and it is known as the vacuum normalization. To fully specify our operators and solve Eq.~\eqref{eomosc}, we need to set initial conditions that will physically determine the annihilation operator and, consequently, the vacuum state of the theory.

\subsection{Adiabatic Vacuum}

In quantum field theory, due to the non-applicability of the Stone-Von Neumann Theorem, different choices of Hilbert Space that are consistent with the commutation relations in Eq.~\eqref{commutation_relations} are not unitary equivalent, which means that they lead to different physical predictions \cite{wald1994quantum}. Therefore, one also needs a prescription to construct the associated Hilbert Space of a quantum field theory.

Although in usual Minkowski space-time one may use its symmetries to define the Hilbert space, in curved spacetimes one needs other techniques to construct said space~\cite{birrell1984quantum}. This problem can be mapped to a choice of operators $\hat{a}_{k}$ that annihilate the vacuum state $\ket{0}$, which in turn depends on the choice of mode functions $\zeta_{k}(t)$ that satisfy the normalization condition in Eq.~\eqref{basec}~\cite{mukhanov2007introduction}. Since such a condition is preserved by the time evolution, it suffices to choose a set of initial conditions $\dcha{\zeta_{k}(t_{0}), \Pi_{k}(t_{0})}$ at an initial time $t_{0}$ \cite{vacuum2022}.

A widely used prescription to define such a vacuum state is known as the adiabatic
vacuum prescription~\cite{birrell1984quantum, mukhanov2007introduction}. Its main idea is
to set initial conditions by demanding that the mode functions $\zeta_{k}(t)$ coincide
with their ${\cal N}$ order adiabatic approximation, $^{({\cal N})}\zeta_{k}(t)$. This
is implemented if one chooses the initial conditions~\cite{birrell1984quantum}
\begin{align}
	\label{v_init}
	{ }^{(\mathcal{N})}	\zeta_k(t_0) & = \frac{1}{\sqrt{m(t_0)\omega_k}(t_0)}e^{i\alpha_k(t_0)}~\text{and}~ { }^{(\mathcal{N})}	\dot{\zeta}_k(t_0) = im(t_0)\omega_k(t_0)\zeta_k(t_0)
	,\end{align}
where $\omega_k^2(t) = \frac{c_s^2 k^2}{\bar{a}^2}$, $m(t) = 2z^2$ and $\alpha_k$ is defined by the relation
\begin{align}
	{ }^{(\mathcal{N})}	\zeta_k(t_0) \dot{\alpha}_k(t_0) & = 1.
\end{align}

Using these initial conditions for the modes, a well-defined vacuum state is obtained,
which is in turn used to construct the Fock Space by successive applications of the
creation operator $\hat{a}^{\dagger}_{k}$ and their linear combinations
\cite{mukhanov2007introduction}. Now that we have defined the appropriate initial
conditions for quantization, we may use them to solve the equations of motion in
Eq.~\eqref{eomosc} and obtain the mode functions, which we shall do numerically.
	{\color{red} Finally, it is important to note that the adiabatic vacuum prescription is
		consistent with the perturbative analysis of the model, as it leads to small
		perturbations as discussed in Ref.~\cite{vitenti2012large}.}


%%%%%%%%%%%%%%%%%%%%%%%%%%%%%%
\subsection{Numerical Solution}

The equations of motion for the curvature perturbation modes are solved numerically using the Numerical Cosmology library~(NumCosmo)~\cite{Vitenti2014}. Specifically, the NcmCSQ1D and the NcHIPertAdiab algorithms are employed for this purpose. The equations are split into two parts: one representing a harmonic oscillator with a mass $m_A$ and the other representing the time evolution of the conjugate momenta $\Pi_{k_A, A}$ of the modes. Explicitly,
\begin{align}
	\Pi_{k_A, A} & = m_A \dot{\zeta}_{k_A, A}
\end{align}
and
\begin{align}
	\dot{\Pi}_{k_A, A} = -m_A\omega^2_A \zeta_{k_A, A}
	,\end{align}
The adiabatic vacuum prescription is considered, and the initial conditions for the modes are set according to Eqs.~\eqref{v_init} up to the fourth order\footnote{This is because the adiabatic approximation leads to an asymptotic series, whose precision drops as one goes up to a certain definite order~\cite{mukhanov2007introduction}. In particular, the code determines the order of optimal precision.}. These initial conditions are used as inputs for the numerical algorithm to compute the Fourier modes of the curvature perturbation modes ${\zeta}_{k_A, A}$ and their conjugate momenta $\Pi_{k_A, A}$. Also, the numerical code computes the validity of the adiabatic approximation for different intervals of time. To assure the veracity of the code, the algorithms have been validated with unit testing and we compared them with analytical solutions (see \href{https://github.com/NumCosmo/NumCosmo/blob/master/tests/test_py_hipert_adiab.py}{GitHub})\footnote{Also, see the \href{https://github.com/NumCosmo/NumCosmo/blob/master/notebooks/primordial_perturbations/single_fluid_qb.ipynb}{Jupyter Notebook} for an example on the usage of the code.}

%%%%%%%%%%%%%%%%%%%%%%%%%%%%%%

\subsection{Spectra}
\label{sec:spectra}
The power spectrum of a theory plays a crucial role in understanding the formation of large-scale structures. The power spectrum also allows one to completely describe the statistics of the problem if the density field is described by Gaussian fluctuations~\cite{Baugh}. To compute the power spectrum in our cosmological model, we first calculate the two-point correlation function of the field variable
$\zeta(\textbf{x},t_i)$ at an initial time $t_i$. This function measures the spatial correlation between fluctuations at different points in space. The correlation function is expressed as an integral over Fourier modes, yielding the desired spatial correlation,  and takes the form
\begin{align}
	\left<\hat{\zeta}(\textbf{x},t_i)\hat{\zeta}(\textbf{y},t_i)\right>
	 & =\frac{1}{(2\pi)^\frac{2}{3}}\int \mathrm{~d}^3{\textbf{k}}\left[ |\zeta_{k}(t_i)|^2 e^{-i\textbf{k}(\textbf{x} - \textbf{y})}\right] \nonumber \\
	 & =\int\frac{ \mathrm{~d}{k}}{k}\left[ P_{\zeta}(k)  \frac{\sin{kR}}{kR} \right]
	,\end{align}
where $R = \abs{\textbf{x} - \textbf{y}}$ and we performed an integral over solid angles in the last equality. In the above, the power spectrum $P_{{\zeta}}$ is defined as\footnote{Note that the power spectrum is already dimensionless, and thus we can use the same definition with our dimensionless variables as well.}
\begin{align}
	\label{eqspec}
	P_{\zeta}(k) & \equiv \frac{k^3|\zeta_k(t_i)|^2}{2\pi^2}
	.\end{align}
Using Eq.~\eqref{eqspec} and the numerical code described in the last section, we can plot the spectra of the theory in Fig.~\ref{fig:figspec}. Once again we have different initial times for the adiabatic limit for distinct modes. We also notice an increasing spectra for all the modes, such that they peak at the bounce time.

As known from the literature, the dust bouncing model has an approximate scale-invariant spectrum, that is
\begin{align}
	\label{powerlaw}
	P_{{\zeta}}(k) & \approx A k^{n_s-1}
	,\end{align}
where $A$ is a constant and $n_s$ is the spectral index. This is a general feature of single barotropic fluid quantum bouncing models, and the spectral index is related to the equation of state parameter $w$ by\footnote{For an explicit semi-analytic derivation, see \cite{nelson_peter_bouncing_original}.}
%
\begin{equation}
	n_{s}(w) = 1 + \frac{ 12w }{ 1 + 3w }\, ,
\end{equation}
%
which is nearly scale invariant for $|w| \ll 1 $. However, the usual positive values of $w$ lead to a blue tilted spectrum $n_{s} > 1$, which differ from CMB observations \cite{planck_inflation_constraints}, consistent with the inflationary scenario prediction of a red tilted spectrum with $n_{s} \approx 0.96$. Although the initial power spectrum is not consistent with observations, one must recall that our model is considering a pure dark matter-dominated universe, neglecting the effects of radiation. In particular, it has been suggested \cite{nelson2021bouncing} that the inclusion of radiation may lead to a red almost scale-invariant spectrum. Therefore, this model must be understood as a first approximation to a more complete model, which is still under development. Furthermore, since most of the modes that influence our universe have crossed the Hubble horizon during dust domination, this means that this model, even with its simplicity, may still cover relevant information for the future complete model.

This result should not advocate for the prediction failure of the contracting scenario
since some works also suggest a blue-tilted spectrum for inflationary
models~\cite{Wang2014, Cai2015}. For instance, in Ref.~\cite{Kuroyanagi2021} they
discuss the possibility of explaining the recent NANOGrav results with a blue-tilted
specrta~\cite{Wu2023}. Also, the spectra are only dictated by a power law for narrow
intervals of momenta, which implies that corrections to Eq.~\eqref{powerlaw} must be
applied. Nonetheless, we see that the contracting phase produces a nearly
scale-invariant spectrum. In the next section, we begin to analyze the primordial black
hole formation seeded by the scalar perturbations in our cosmological background.

The resulting plots in Fig.~\ref{fig:figspec} depict the power spectra for a single momentum mode ($k_a = 0.1$) of the curvature perturbation, the Bardeen field variable, the density contrast, and the evolution of the scale factor over time. For visualization purposes, we use the same time parameter as the one in the numerical code, namely,
\begin{align}
	\label{tauparameter}
	\cosh(\tau_a)^{2} & \equiv \left[1 + \left(\frac{t}{t_b}\right)^2\right]
	,\end{align}
such that $\tau_a$ is dimensionless by definition. The plots show an increasing power spectrum for all fields, peaking at the bounce. Note that $\Phi$ has highly divergent spectra, as discussed previously,  which shows the inapplicability
of the Newtonian Gauge in this model. Additionally, the scale factor decreases with time until reaching its minimum value at the bounce.

\begin{figure}[tbp]
	\centering
	\includegraphics[width=.6\textwidth]{powspec.pdf}
	\caption{Plot of the power spectrum of the curvature perturbation modes $\zeta_k$, the Bardeen field variable $\Phi_k$ and the density contrast $\delta$ computed with the NumCosmo library for $w = 10^{-10}$ and $x_b = 10^{35}$. The red plot represents the scale factor evolution in time. The time parameter in the x-axis is given by Eq.~\eqref{tauparameter}.}
	\label{fig:figspec}
\end{figure}                                                  

\newpage

%%%%%%%%%%%%%%%%%%%%%%%%%%%%%%%%%%%%%%%%%%%%%%%%%%%%%%%%%%
\section{PBH formation during the contracting phase}
\label{sec:formation}

Various mechanisms can result in Primordial black hole formation (see Ref.~\cite{Escriva2023} for a review). In this work, we focus on investigating the formation of primordial black holes through the critical collapse of matter perturbations. This approach has been extensively studied since its popularization by Carr and Hawking~\cite{Hawking1971, Carr1974}, and it has found broad applications in the context of inflationary models. We want to extend this approach to the context of a quantum dust-bouncing model and compute the PBH abundance in this scenario. It is important to note that although we perform all the calculations with the gauge invariant variables, we will analyze our results for the physical density perturbation in the fluid's gauge, as defined in App.~\ref{appgauge}.

\subsection{Formation Criteria}

In our context, critical collapse is the collapse of matter perturbations when they achieve a given threshold. The energy-density perturbation is characterized by the density contrast $\delta$ defined in Eq.~\eqref{densitycon}. We assume that if $\delta > \delta_c$, where $\delta_c$ is the critical threshold, the perturbations will collapse into a black hole. This critical threshold relies both on the cosmological model and the shape of perturbations (see Ref.~\cite{Niemeyer1998, Musco2019} for a comprehensive analysis of this effect) and therefore must be carefully studied as it will heavily impact the formation of black holes.

The critical collapse model assumes the existence of a region of radius $r$ with an overdensity $\delta$. To compute the probability of the existence of this overdense region, we will need to compute the variance of our random variable $\delta$ and instead of working with the original density contrast, we work with its filtered version $\delta_r$, which is defined by
\begin{align}
	\delta_r(t) & =  \int d^{3}\textbf{x}^\prime \delta(t, \vec{x}^\prime) W_r\left(\vec{x} - \vec{x^\prime} \right)
	,\end{align}
where $r$ is a smoothing scale in comoving units related to the mean density by $M = \frac{4\pi r^3 \bar{\rho}}{3}$ and $W$ is the top-hat filter
\begin{align}
	\label{windowf}
	W_r(\vec{x}) & =W_r\left(|\vec{x}| = x\right) =
	\begin{cases}
		\frac{1}{4\pi r^3} & \text{if $x \leq r$} \\
		0                  & \text{otherwise}
	\end{cases}
	.\end{align}
The filter function is used to select a desired scale such that we would only analyze collapsed objects that lie in this interval\footnote{Frequently we will usually refer to such scale in terms of the mass $M$ instead of the comoving radius $r$ that encloses this mass.}. In this work, we are using a simple top-hat windowed function for the filter. We usually work on the Fourier space with comoving wave-number $k$, such that the filter is given by
\begin{align}
	W_r(k) & = \frac{3}{(kr)^2}\left(\frac{\sin(kr)}{kr} - \cos(kr)\right) = \frac{3}{kr}j_1(kr)
	,\end{align}
where $j_1$ is the spherical Bessel function of the first kind. We will see that the filter is necessary to analyze scales where the PBH formation is most likely to happen.

Coming back to $\delta_r$, we know from our quantum fields that the density contrast is a random field following a Gaussian distribution with a zero mean. This is also supported by significant research on the statistics of peaks in random Gaussian fields, which can give rise to collapsed objects, as explored in Ref.~\cite{Bardeen1986statistics}. Mathematically, the probability of a region with radius $r$ having an overdensity $\delta_i$ is given by
\begin{align}
	\label{gaussdelta}
	P(\delta_i) & = \frac{1}{\sqrt{2\pi} \sigma_r} \exp(-\frac{\delta^2_i}{2 \sigma_r^2})
	.\end{align}
This Gaussian is completely defined by its variance
\begin{align}
	\label{eqsigmar}
	\sigma_r^2 & =\left< \delta_r^2(\vec{x}) \right> \equiv \sigma_r^2 = \frac{1}{2\pi^2} \int_0^\infty \frac{ \mathrm{~d}{k}}{k}\left[ P_{\tilde{\delta}}(k) \abs{W_r(k)}^2\right]
	,\end{align}
where $P_{\tilde{\delta}}(k)$ is given by Eq.~\eqref{eqspec} with the density contrast modes. Unfortunately, for our bounce model, this integral does not converge. Before we perform this calculation, we will analyze the critical threshold for PBH formation, which will lead to scale constraints that will help us solve this problem.
%%%%%%%%%%%%%%%%%%%%%%%%%%%%%%%%
\subsection{Critical Threshold}

\label{critical_delta}
In inflationary models, the frozen super-Hubble density perturbations re-enter the Hubble horizon at the end of the potential decay and collapse into a black hole if they have values above the critical threshold~\cite{Martin2014}. Hence the Hubble horizon is viewed as a characteristic scale for their formation and the threshold ($\delta_c$) is obtained via the Minster-Sharp equations (see Ref.~\cite{Musco2019}). However, in the bounce scenario, the perturbations constantly evolve in time and there is not only one characteristic scale for the formation, since any perturbation above a critical threshold may collapse as they are not frozen. Thus, we must carefully analyze how to obtain this critical value starting with a local metric for the collapse.

In App.~\ref{appc}, we find the solutions for Einstein's equations of a critical collapse represented with a local metric that depends on the local function $R(t,r)$, which led to the LTB set of solutions
\begin{align}
	\label{ltb3a}
	R(\theta, r) & = \frac{r (1+\delta_{ini})}{\delta_{ini}}\sin^2\left(\frac{\theta}{2}\right),                                        \\
	\label{ltb4a}
	t(\theta,r ) & = t_1(r) + \frac{1 + \delta_{ini}}{2\bar{H}_{ini}\delta^{\frac{3}{2}}_{ini}}\left(\theta - \pi - \sin \theta \right)
	,\end{align}
such that we use the notation $\delta_{ini} = \delta(t_{ini})$ at the initial collapse time. In this section, we wish to impose some constraints on those solutions to find the exact point at which the perturbations will form a black hole. Consequently, this will lead to a critical value for the density contrast. First of all, let us analyze the parameter $\theta$. From Eq.~\eqref{sintheta},
\begin{align}
	\label{thetadef}
	\theta_{ini} & = \pm 2\arcsin(\sqrt{\frac{\delta_{ini}}{1+\delta_{ini}}})
	.\end{align}
To choose between the positive or negative value, we use our initial condition in Eq.~\eqref{hinicond} and Eq.~\eqref{hdefltb} as a test, such that replacing Eq.~\eqref{thetadef}
\begin{align}
	\left.\frac{\dot{R}}{R}\right|_{ini} & =  \left.\frac{\frac{\partial R}{\partial \theta}}{R \frac{\partial t}{\partial \theta}}\right|_{ini} = \pm \bar{H}_{ini}
	.\end{align}
From the above, we note that the positive sign in Eq.~\eqref{thetadef} leads to the right definition of our initial conditions. If one chooses the negative sign, and as $\bar{H}_{ini}$ is negative in the contracting phase, we would end up with an initially expending patch. This feature can be seen in Fig.~\ref{theta_evol}, where Eqs.~\eqref{ltb3a} and \eqref{ltb4a} were solved analytically with Wolfram Mathematica~\cite{Mathematica}. The blue lines represent the positive choice that leads to the LTB collapse model in a contracting universe, while the orange lines describe an LTB collapse model with an initially expanding patch. On the left side of the figure, for a larger value of $\delta_{ini}$, we see that both solutions are similar as the orange plot has an insignificant initial expansion. For a smaller value of $\delta_{ini}$ on the right, the model in orange first goes on an expansion phase followed by the collapse, while the blue plot is already in a contracting phase and collapses much earlier.

\begin{figure}[tbp]
	\centering
	\includegraphics[width=.8\textwidth]{mathematica_theta_evol.pdf}
	\caption{Plot of $R$ vs $\theta$. We have chosen $\bar{H}_{ini} = -1$ for simplification purposes, which implies that $t$ is in units of $1/\bar{H}_{ini}$. On the left side, we see the case for a large initial density contrast while the right side indicates a small initial density contrast. The blue and the orange lines indicate the positive and negative initial conditions for $\theta_{ini}$ respectively. The blue line represents the right choice and a contracting model while the orange line indicates an expansion model. We chose different values of $\delta_{ini}$ for both graphs as the difference between both choices becomes more evident for a smaller initial density contrast. The data was computed with Wolfram Mathematica~\cite{Mathematica} and the graphs were generated in Python.}
	\label{theta_evol}
\end{figure}


Let us now analyze the PBH formation. The collapsed object will form a singularity when $\frac{\partial R}{\partial r} = {R}^\prime = 0$~\cite{Dey2023}. From Eq.~\eqref{ltb3a} we see that
\begin{align}
	{R}^\prime & = \frac{ (1+\delta_{ini})}{\delta_{ini}}\sin^2\left(\frac{\theta}{2}\right) = 0
\end{align}
if $\theta = 0$. Consequently, we want to analyze the necessary time for a black hole to form, which takes place at $\theta = 0$. From Eq.~\eqref{ltb4a} we have that the final time $t_f$, i.e., the time at formation is
\begin{align}
	t_f & = t(0,r ) =t_1(r) - \frac{\pi(1 + \delta_{ini})}{2\bar{H}_{ini}\delta^{\frac{3}{2}}(t_{ini})}
	.\end{align}
We want to compute the formation time $\Delta t = t_f - t_i$, such that $t_i = t(\theta_{ini}, r)$ with
\begin{align}
	\theta_{ini} & = 2\arcsin(\sqrt{\frac{\delta_{ini}}{1+\delta_{ini}}})
	.\end{align}
Hence, from Eq.~\eqref{ltb4a}, we have that
\begin{align}
	\label{criticallinear}
	\Delta t & = t_f - t_i \nonumber                                                                                                                                                                                 \\
	         & =\frac{(1 + \delta_{ini})}{2\bar{H}_{ini}\delta^{\frac{3}{2}}_{ini} } \left(  -\theta_{ini} +\sin{\theta_{ini}}\right) ,\nonumber                                                                     \\
	         & =\frac{(1 + \delta_{ini})}{2\bar{H}_{ini}\delta^{\frac{3}{2}}_{ini} } \left( -2\arcsin(\sqrt{\frac{\delta_{ini}}{1+\delta_{ini}}}) +\sin(2\arcsin(\sqrt{\frac{\delta_{ini}}{1+\delta_{ini}}}))\right)
	.\end{align}
Assuming that $\delta_{ini} \ll 1$, we can expand the above relation such that
\begin{align}
	\label{deltasmall}
	\Delta t & = -\frac{2}{3\bar{H}_{ini}}+\frac{2\delta_{ini}}{15\bar{H}_{ini}} + \mathcal{O}^2(\delta_{ini})
	.\end{align}
We now know how to compute the time interval with the right side of the above equation. We shall now study the left side of Eq.~\eqref{deltasmall} as we want to find a constraint on the final formation time $t_f$. Supposing that the PBH is formed way before the bounce\footnote{We do not make this approximation in the numerical code. This approximation is only used to demonstrate our calculations.}, we can approximate Eq.~\eqref{bohm_hubble} to
\begin{align}
	\label{hubblefarbounce}
	\Bar{H} & = \frac{2}{3t}
	.\end{align}
Plugging this result for $t_i$ on the left side of Eq.~\eqref{deltasmall} leads to
\begin{align}
	\label{tf}
	t_f & = \frac{2\delta_{ini}}{15\bar{H}_{ini}}
	.\end{align}
Let us rewrite the above expression in terms of the redshift function $x$ defined as
\begin{align}
	x(t) & \equiv  \frac{\bar{a}_0}{\bar{a}(t)}
	,\end{align}
such that $\bar{a}_0$ is the scale factor today and the Hubble function can be written as
\begin{align}
	\label{Hx}
	\bar{H} & = -\bar{H}_0\sqrt{\Omega_w} x^{3/2}
	,\end{align}
being $\Omega_w$ the dust density in the universe today and thus Eq.~\eqref{tf} becomes
\begin{align}
	\label{tfx}
	t_f & = -\frac{2 \delta_{ini}}{15\bar{H}_0 \sqrt{\Omega_w} x_{ini}^{3/2}}
	.\end{align}

We need to impose a limit on the formation time to find the critical values of delta. For collapsed objects larger than the Hubble radius, their dynamics will be dominated by the FLRW metric and not the LTB approximation. The supper-Hubble perturbations will be frozen and thus there is no collapse for these scales. This sets the Hubble length as an upper cut-off, i.e., their formation time must be at most the time for the perturbation to achieve the Hubble radius, labeled as $t_H$.  The radius is related to the perturbation's wavelength $\lambda$ and comoving wavenumber $k$ by~\cite{Quintin2016}
\begin{align}
	\label{radius}
	r       & =\frac{\lambda}{2}, \\
	\label{wavelength}
	\lambda & = \frac{2 \pi}{k}.
\end{align}
We want to analyze when a perturbation with wavenumber $k$ has the same size as the Hubble radius, that is,
\begin{align}
	\label{hubblek}
	k_H & = \frac{1}{x}\abs{\Bar{H}}
	.\end{align}
From Eq.~\eqref{hubblek} and Eq.~\eqref{Hx}, we have that the redshift function associated with this scale is given by
\begin{align}
	\label{xh}
	x_H & = \frac{k_a^2 }{\Omega_w}
	,\end{align}
where $k_a$ is the comoving dimensionless wave number and the subscript $H$ indicates the Hubble scale. In terms of time, we can rewrite Eq.~\eqref{xh} with Eq.~\eqref{hubblefarbounce} and Eq.~\eqref{Hx} such that
\begin{align}
	t_H & = -\frac{2\Omega_w}{3 \bar{H}_0 k_a^2}
	.\end{align}
Finally, we must ensure that
\begin{align}
	\label{deltacf}
	t_f \leq t_H ~\text{or}\nonumber \\
	\delta_{ini} \geq \frac{5\Omega_w^{\frac{3}{2}} x_{ini}^{\frac{3}{2}}}{k_a^3}
	,\end{align}
where we used Eq.~\eqref{tfx} and Eq.~\eqref{kadm}. From the above, we can see that the critical threshold depends both on the scale and time and it can be set from the saturation of the inequality as
\begin{align}
	\label{deltacfinal}
	\delta_c & =
	\frac{5 \Omega_w^{\frac{3}{2}} x_{ini}^{\frac{3}{2}}}{k_a^3}
	.\end{align}
Let us now evaluate two cases: one considering the dark matter as a pressureless fluid and another considering really small but finite pressure.

\subsection{$\Bar{p} = w \bar{\rho}$}

In the presence of pressure, it is well known that the Jean's length determines the
smaller scale sufficient for a black hole to be formed. The pressure forces oppose the
collapse and only for scales above this limit a black hole can be formed. Thus, we are
interested in the physical radius in Super-Jeans/Sub-Hubble scales, that is,
$r_j<r<r_H$. The Jeans comoving scale $k_j$ is given by~\cite{Quintin2016}
\begin{align}
	\label{jeansk}
	k_J & = \sqrt{\frac{3}{2}}\frac{1}{x} \frac{\abs{\Bar{H}}}{c_s}, \\
	.\end{align}
Thus, no structure smaller than the Jeans length can collapse and we have a lower momenta cutoff $k<k_j$. Plugging this requirement plus our upper limit in Eq.~\eqref{jeansk} leads to
\begin{align}
	\label{jeanslimit}
	k_H<k                                                                            & \leq k_j~\text{or}\nonumber                                     \\
	\left(\frac{1}{\sqrt{\Omega_w}}\right)^{\frac{3}{2}}>\frac{x^{\frac{2}{3}}}{k^3} & \geq \left(\frac{2c_s^2}{3\sqrt{\Omega_w}}\right)^{\frac{3}{2}}
\end{align}
where we have used Eq.~\eqref{hubblefarbounce} and Eq.~\eqref{Hx}. The above relation indicates that for scales/times that do not satisfy the inequality, the perturbation modes do not contribute to PBH formation. Hence we must evaluate the collapse starting between the Jeans-Hubble time and always ending when the perturbations reach the Hubble length. We can see in Fig.~\ref{jeans_hubble_time} the respective Jeans and Hubble time for each different scale. Note that the interval between both times is always constant for all modes, as it depends only on the dark matter equation of state $w$. Thus, for smaller $w$, the gap between the times increases and the PBH formation is enhanced as there is more time for them to be formed.

\begin{figure}[tbp]
	\centering
	\includegraphics[width=.6\textwidth]{jeans_hubble_time_vs_k.pdf}
	\caption{ Plot of the Jeans and Hubble time computed with the NumCosmo library for $w = 10^{-10}$ and $x_b = 10^{35}$. These times correspond to when the matter density perturbations reach either the Jeans or the Hubble length respectively for each mode $k$. The time parameter in the y-axis is given by Eq.~\eqref{tauparameter}.}
	\label{jeans_hubble_time}
\end{figure}
\newpage

Finally, we have the critical threshold in Eq.~\eqref{deltacfinal} with Eq.~\eqref{jeanslimit} corresponding to the allowed scales. Hence, not all perturbations will collapse into primordial black holes. Note that for smaller values of $c_s^2$, there is more time between the Jeans and the Hubble scale, which allows for more perturbations to collapse. Thus the abundance of PBH in this model is directly related to the equation of state of dark matter, as we will see in the next section.

In this context, let us now analyze Eq.~\eqref{deltacfinal} for the super-Jeans/Sub-Hubble scales, depicted in Fig.~\ref{deltacfig}. Each plot begins for $\delta_c$ being computed for $t_i = t_j$ and goes until $t_i = t_H$. The first case corresponds to the collapse starting when the modes just achieved the Jeans length and thus have the maximum amount of time to collapse until reaching the Hubble length. The end of the time interval corresponds to a collapse taking place right before the perturbation reaches the Hubble length, which leads to a maximum value of the critical threshold since there is a minimum amount of time for the collapse of the perturbations. We can see that all plots go from $\delta_c \approx 10^{-14}$ at Jean crossing time to $\delta_c \approx 5.0$ at the Hubble scale. However, the closer we get to the Hubble time as the initial time, the worse the approximation $\delta_c \ll 1$ gets and thus this should be disregarded. If $t_i = t_H$, we have that $\delta_c \sim \infty$, since there is no time for the collapse to happen. In Fig.~\ref{theta_evol} we have solved Eq.~\eqref{deltacfinal} analytically for larger values of $\delta$. Nevertheless, such values of density contrast will never be achieved during the contracting phase.

\begin{figure}[tbp]
	\centering
	\includegraphics[width=.6\textwidth]{deltac_jh_vs_tau.pdf}
	\caption{ Plot of the critical threshold vs time computed with the NumCosmo library for $w = 10^{-10}$ and $x_b = 10^{35}$. Each plot refers to a different scale $k$, while points on the same plot represents a different initial time for the collapse. The time interval for each mode starts at the Jeans scale, leading to the smallest threshold values, and ends for initial times when the perturbations reach the Hubble scale, which leads to higher threshold values. The time parameter in the x-axis is given by Eq.~\eqref{tauparameter}.}
	\label{deltacfig}
\end{figure}
\newpage
\subsection{$\bar{p} = w \bar{\rho}, \quad w \ll 1$}
\textcolor{red}{We want to analyze what would happen if the dark matter had such a small equation of state that could be thought as a presureless fluid.} In this case, Eq.~\eqref{deltacf} tells us that, for every scale $k$, we can always find a sufficient time in the past such that $x_{ini} \ll 1$, leading to the collapse of all density perturbations. In other words, if the universe is old enough, all perturbations will eventually collapse into a black hole until the universe reaches the bounce. If there was no bounce, the universe would completely collapse into black holes and our model would be highly unstable.

It is indeed expected that a fluid with no pressure leads to a total collapse since no forces are opposing the collapse and dust only interacts via gravity. Thus all dark matter that we see today would indeed be contained in primordial black holes. However, this hypothesis was already refuted by some works by constraining the PBH abundance in dark matter based on observational effects (see Refs.~\cite{Villanueva2021, Carr2021}). Hence we must consider a dust fluid with a non-vanishing equation of state.
\subsection{Filtered Variance}
Let us now compute the variance which will determine the distribution for PBH. Since sub-Jeans and super-Hubble scales do not contribute to the PBH formation that we are interested in, we want to compute the integral in  Eq.~\eqref{eqsigmar} between $k_H$ and $k_j$. To do so, we first need to evaluate $P_{\tilde{\delta}}(k)$ using Eqs.~\eqref{deltarho}-\eqref{phiz} and the field expansions in Eqs.~\eqref{vexp} and~\eqref{pvexp}. Explicitly,
\begin{align}
	\label{psiq}
	\hat{\Psi}(\eta,\textbf{x}) & =\int\frac{\mathrm{d}^3\textbf{k}}{(2\pi)^{\frac{3}{2}}}
	\frac{\bar{H}}{2\Bar{a}k^2}\left[\Pi_{\zeta_{k}} e^{i\textbf{k}\textbf{x}} a_{k} +
	\Pi_{\zeta_{k}}^{ *}e^{-i\textbf{k}\textbf{x}} a_{k}^{\dagger}\right]
	,\end{align}
\begin{align}
	\bar{D}^2\hat{\Psi}(\eta,\textbf{x}) & =
	-\int\frac{\mathrm{d}^3\textbf{k}}{(2\pi)^{\frac{3}{2}}}
	\frac{\bar{H}}{2a^3 }\left[\Pi_{\zeta_{k}} e^{i\textbf{k}\textbf{x}} a_{k} +
	\Pi_{\zeta_{k}}^{*}e^{-i\textbf{k}\textbf{x}} a_{k}^{\dagger}\right]
\end{align}
and
\begin{align}
	\hat{\tilde{\delta\rho}}(\textbf{x}) & =
	\int_{-\infty}^{\infty}\frac{\mathrm{~d}^3\textbf{k}}{(2\pi)^{\frac{3}{2}}}
	\left\{ \left(\frac{\bar{H}}{\kappa \bar{a}^3}\right)\Pi_{\zeta_{k}} e^{i\textbf{k}\textbf{x}} a_{k} +\right. \nonumber                                     \\
	                                     & \left. \left(\frac{\bar{H}}{\kappa a^3}\right)\Pi_{\zeta_{k}}^{*} e^{-i\textbf{k}\textbf{x}} a_{k}^{\dagger}\right\}
	.\end{align}
Thus, the two-point function and the variance are given by
\begin{align}
	\label{2point}
	\left<\hat{\Tilde{\delta}}(\textbf{x})\hat{\Tilde{\delta}}(\textbf{y})\right> & =
	\frac{1}{2\pi^2} \int_0^\infty \frac{\mathrm{~d}k}{k}~\left[P_{\tilde{\delta}}(k)
		\frac{\sin{kR}}{kR} \right]
\end{align}
and
\begin{align}
	P_{\tilde{\delta}}(k) & = \abs{\Pi_{\zeta_{k}}}^2\left( \frac{1}{9\bar{a}^6 \Bar{H}^2(1+w)^2}\right),
\end{align}
where we used the flat Friedmann equation
\begin{align}
	\bar{H}^2 = \frac{\kappa \Bar{\rho}}{3}.
\end{align}
We are now able to compute the variance modes numerically using the above relation together with Eq.~\eqref{physicalrho} in the fluid gauge and our numerical code.

\begin{figure}[tbp]
	\centering
	\includegraphics[width=.6\textwidth]{sigmas.pdf}
	\caption{ Plot of the filtered variance computed with the NumCosmo library for $w = 10^{-10}$ and $x_b = 10^{35}$. Each point corresponds to the variance computed when the perturbations become either the Jeans (blue plot) or the Hubble (orange plot) size for each scale.}
	\label{sigmas}
	.\end{figure}
\newpage

The values for the filtered variance computed with Eq.~\eqref{eqsigmar} from $k_H$ to $k_j$ are displayed in Fig.~\ref{sigmas}. We have plotted the values of $\sigma_r$ for two different times versus radius. The sub-indexes $H$ and $j$ indicate that we are computing the variance at a time when the perturbations are the size of the Jeans length or the Hubble horizon respectively. We can see a linear relation between the filtered variance and the scale radius, such that the former decays with the latter. Larger values of variance indicate a broader Gaussian distribution, which enhances the PBH formation. Thus, from this figure, we see that the formation of PBHs on smaller scales is prioritized rather than on larger scales. We now shall compute the PBH abundance in this model.


%%%%%%%%%%%%%%%%%%%%%%%%%%%%%%%%%%%%%%%%%%%%%%%%%%%%%%%
\subsection{Mass Function}
\label{psf}

To analyze the abundance of collapsed objects, we use the Press-Schechter (PS) formalism, first proposed in~\cite{press1974}. This ansatz presumes that the objects are formed through a nonlinear collapse and have the property of being universal regarding different cosmological models. Other works have proposed specific mass functions for the primordial black hole production for specific mass ranges, for example~\cite{Cai2023, Bi2024}. However, in this work, we restrain ourselves to the universal PS formalism to analyze all possible scales. Following this approach, the density of collapsed objects per mass scale is given by the mass function
\begin{align}
	\label{ps}
	\frac{dn(z, M)}{dM} & = -\frac{\bar{\rho}_m(z)}{M}\frac{d\beta(z,M)}{dM}
	,\end{align}
where $\frac{dn(M,z)}{dM}$ is the mass function, $z$ is the object's redshift, $M$ is the object's mass, and $\beta$ is the fraction of collapsed objects in the mass range of $M+dM$.

Assuming that the objects will collapse when $\delta  > \delta_c$, we can measure the mass fraction inside spheres of radius $r$ for a time $t$ that is constituted of collapsed objects as
\begin{align}
	\label{betaf}
	\beta(t, r) \equiv \frac{\rho_{PBH}}{\rho} & =\int_{\delta_c}^{\infty}\mathrm{d}\delta~P(\delta_r) \nonumber \\
	                                           & = \text{erfc}\left(\frac{\delta_c}{\sqrt{2}\sigma_r(t)}\right)
	,\end{align}
where $erfc$ is the complementary error function. However, $\beta$ is a probability density of how likely are perturbations to collapse after a time $t$. For instance, if we compute it at the time when the perturbations reach the Hubble size, $\delta_c = \infty$, this integral will vanish, denoting that there are no more PBHs to be formed after this time. In our context, we are interested in analyzing how many PBHs were formed in the whole contracting phase so we can compute its abundance today. This function shall be given by
\begin{align}
	\label{densityf}
	F(t, r)_{PBH} & \equiv \max(\beta(t_i,r)),~\text{for}~t_i \leq t
	.\end{align}
The maximum of the beta function indicates for which time there is a higher probability of PBHs to be formed, which will be given by the Jeans time $t_{j}$. Thus $F(t, r)$ with $t_i = t_j$ denotes the actual mass fraction of PBHs formed during the entire bouncing phase. Hence, as an approximation, we will compute the mass fraction for the maximum probability of forming PBHs, i.e., always beginning at $t_j$, to find the abundance of PBHs today.

Another crucial quantity is the abundance of collapsed objects in the universe $\Omega_{PBH}$, defined in Ref.~\cite{Carr2016} as the ratio between the density of PBHs and the background universe's density today. This parameter will serve as a guide to compare our theoretical predictions with observational data~\cite{Villanueva2021}. Since both the background and the PBHs are formed by dust, both grow with the same power of the scale factor and thus the PBH abundance today will be given by
\begin{align}
	\Omega_{PBH} & = F(t, r) \Omega_{DM} = \text{erfc}\left(\frac{\delta_c}{\sqrt{2}\sigma_r(t_{j})}\right) \Omega_{DM}
	.\end{align}
In Fig.~\ref{fvsm} we can see the values of the density function in Eq.~\eqref{densityf} for different mass scales. Keep in mind that everything is being computed for the collapse starting at the Jeans time for every scale.

\begin{figure}[tbp]
	\centering
	\includegraphics[width=.6\textwidth]{F.pdf}
	\caption{ Plot of the mass fraction of PBHs forming during bounce versus their formation mass in solar masses for $w = 10^{-10}$ and $x_b = 10^{35}$. The computations were done using the NumCosmo library.}
	\label{fvsm}
\end{figure}

The complementary error function in Eq.~\eqref{betaf} has higher values when the argument is closer to one, i.e., when the filtered variance and the critical density contrast are in the same scale. We can see in Fig.~\ref{fvsm} that this is only true for small mass scales, where the density function is equal to one. However, at these mass scales, every formed PBH black hole would eventually evaporate. From Ref.~\cite{Villanueva2021}, the evaporation constraint determines that only black holes with masses $M>10^{-18}M_{\odot}$ would have not completely evaporated today. Thus, for larger scales, the density fraction equals zero, and there is no relevant PBH formation in this model. Nonetheless, since we are dealing only with dust, which has a small and not completely determined equation of state, let us analyze the same function for different values of $w$, depicted in Fig.~\ref{fvsw}.

\begin{figure}[tbp]
	\centering
	\includegraphics[width=.6\textwidth]{F2.pdf}
	\caption{ Plot of the mass fraction of PBHs forming during bounce versus their formation mass in solar masses for different value of $w$ and $x_b = 10^{35}$. The computations were done using the NumCosmo library.}
	\label{fvsw}
\end{figure}

For smaller values for the equation of state of cold dark matter, we can see an enhancement in the density function.  This case reflects our previous discussion of a pressureless fluid. As we go further into the smaller values, almost every scale will collapse into a black hole and thus the density function equals one.


%%%%%%%%%%%%%%%%%%%%%%%%%%%%%%%%%%%%%%%%%%%%%%%%%%%%%%%%%%

%%%%%%%%%%%%%%%%%%%%%%%%%%%%%%%%%%%%%%%%%%%%%%%%%%%%%%%%%%
\section{Discussion and Conclusions}
\label{sec:discussion}
In this study, we investigated the formation of PBHs in a flat quantum bouncing model containing only dark matter with a small equation of state parameter $w$. Despite the spectra from this model exhibiting a slight blue tilt, which deviates from observations of the Cosmic Microwave Background (CMB), this discrepancy should not preclude the model, as the inclusion of radiation is expected to induce a different spectrum that may be compatible with observations \cite{Vitenti2012}. The dark matter-only model serves as an initial attempt to quantify PBH formation during the contracting phase.

Constraints for PBH formation were established, requiring their lengths to fall between the Jeans and Hubble scales. These upper and lower bounds enabled the computation of the critical density contrast, detailed in Sec.~\ref{critical_delta} and App.~\ref{appc}. We obtained a time/scale-dependent critical threshold different from the usual constant due to the contracting dynamics. This behavior was expected as there is not only one characteristic scale for the collapse in the bounce model since the perturbations may collapse for any scale between the Jeans and the Hubble scale, which affects the threshold calculation. For the case of a pressureless field, perturbations for all scales may collapse and the bounce/contracting phase duration would be the only constraint for PBH formation.

For a small but non-zero pressure, we computed the mass fraction of primordial black holes in the universe today, given by Eq.~\eqref{densityf}. In Fig.~\ref{fvsm}, for a fixed value of $w$, we can see an enhancement of the density function for smaller mass scales of PBHs. However, for $w = 10^{-10}$, the formed primordial black holes would have such small masses at smaller scales that they would have evaporated completely today. Figure~\ref{fig:pbhdm} was generated in Ref.~\cite{Villanueva2021} by imposing observational constraints on the mass fraction of PBHs. They considered a range of effects such as black hole evaporation, microlensing, gravitational wave measurements, and others. The white regions of the graph correspond to accepted values of the mass fraction and the colored represent physical observational effects that exclude the possibility of PBHs constituting DM in that mass range. We can see that PBHs under the mass range $M <  10^{-18}M_\odot$ are disregarded due to evaporation, which implies that the formed PBHs depicted in Fig.~\ref{fvsm} would have completely evaporated today.

\begin{figure}[tbp]
	\centering
	\includegraphics[width=1.0\linewidth]{pbh_dm}
	\caption{Plot of the density function equivalent to the one in Eq.~\eqref{densityf} versus the formation mass of PBHs. The colored areas are related to Constraints on the PBH fraction constituting dark matter today. Each color corresponds to a different probe considered to constrain PBH abundance. The uncolored area of the graph corresponds to the possible fraction values that agree with all probes. Figure from Ref.~\cite{Villanueva2021}.}
	\label{fig:pbhdm}
\end{figure}

Still on Fig.~\ref{fvsm}, for larger scales, there is no significant formation as the time interval for the collapse is not sufficient nor is the variance large enough. Also, the variance decreases with scale, as seen in Fig.~\ref{sigmas}, while the critical threshold remains constant as it is always computed for the Jeans time at the given scale. Hence the complementary error function decreases since the argument grows and the PBH fraction becomes insignificant.

We have seen in Eq.~\eqref{jeanslimit} that the time interval for which the perturbations become super-Jeans/sub-Hubble is proportional to $c_s^3 = w^{\frac{3}{2}}$. Thus smaller values for the equation of state allow for more time for the perturbations to collapse and therefore enhance the primordial black hole formation. In Fig.~\ref{fvsw}, we see that only models featuring sufficiently small values for the equation of state of dark matter ($w < 10^{-17}$) may lead to a non-vanishing mass fraction of primordial black hole at relevant scales ($M >  10^{-18}M_\odot$). If we go even further on small values of $w$, more scales start to collapse as we have $F\sim 1$ for all scales as we approach the pressureless fluid scenario.

Another approach for dark matter would be to consider a non-zero temperature. In Ref~\cite{Armendariz2014} dark matter is treated as a non-relativistic gas and, even for cases where $w\ll1$, the non-vanishing temperature would erase structure formation for scales smaller than the corresponding free-streaming length, which once again would contribute to decrease PBH formation in this model. If we treat DM as a relativistic gas, its temperature grows with $\Bar{a}^{-2}$~\cite{Mukhanov1992}. Thus, its temperature and pressure would diverge close to the bounce and prevent the collapse and PBH formation. We conclude that the formation of PBHs in the flat-dust bounce in observable scales is improbable as the analysis is robust against the formation on larger scales, leading to an insignificant fraction of PBHs as DM today.

\textcolor{red}{In this work, we have investigated the formation of primordial black
	holes within a Friedmann background, incorporating scalar perturbations. An alternative
	approach would be to explore PBH formation in a homogeneous yet anisotropic background.
	The presence of anisotropy could give rise to PBHs with significant angular momentum,
	contrasting with those formed in standard inflationary scenarios. A study of this
	possibility is left for future work.}. The next phase of this research would be to apply
the same methodology to a universe populated with radiation and dark matter, which more
closely resembles the physical universe and the interaction between both fluids may
affect PBH production. If such a model proves reasonable, it would also be interesting
to improve the computation of the LTB model in App.~\ref{appc} for larger values of the
equation of state. Despite the Jeans scale being a good approximation, the introduction
of a non-pressureless fluid would also require the study of the pressure impact in our
characteristic scales. Furthermore, in a radiation-dominated quantum bouncing model, one
expects a greater blue tilt on the spectral index, which may affect the density contrast
modes for smaller scales. Said effect could lead to a larger formation of PBHs. This
analysis is left for future work.


%%%%%%%%%%%%%%%%%%%%%%%%%%%%%%%%%%%%%%%%%%%%%%%%%%%%%%%%%%

%%%%%%%%%%%%%%%%  Appendices  %%%%%%%%%%%%%%%%%%%%%%%%%%%

\appendix

\section{Einstein Equations for Spherical Collapse}
\label{appc}

In this Appendix, we are interested in solving the Einstein equations for a metric representing the spherical collapse. We intend to use these solutions to compute the critical threshold in Sec.~\ref{critical_delta}. We adapted the calculations initially done in Ref.~\cite{Gonccalves2000} and corrected by Ref.~\cite{Martin2020}. In these references, a scalar field is used as the source for the energy-momentum tensor. We shall now perform the same calculations for a pressureless barotropic fluid.

An inhomogeneous but spherically symmetric space can be represented by the metric\footnote{This metric can also be written with a lapse function that shifts the time component. However, one can always redefine the coordinate to incorporate this quantity.}
\begin{align}
	\mathrm{d} s^2=-\mathrm{d}t^2+e^{-2 \Lambda(t, r)} \mathrm{d} r^2+R^2(t, r)\left(\mathrm{d} \theta^2+\sin ^2 \theta \mathrm{d} \varphi^2\right)
	,\end{align}
where $\Lambda(r,t)$ and $R(t,r)$ are functions of the local coordinates to be defined and we are considering spherical coordinates. Since none of these functions do not depend on the angular variables, this metric may represent any type of spherical collapse. To find the local functions that describe our problem, we need to solve Einstein's equations for an energy-momentum tensor that corresponds to our symmetries. We shall represent $\partial_r \equiv {}^\prime$.

Following \cite{Martin2020}, the Einstein's tensor components are
\begin{align}
	G_{t t}           & =\frac{1}{R^2}\left[1+\dot{R}^2-2 \dot{\Lambda} \dot{R} R-R e^{2 \Lambda}\left(2 \Lambda^{\prime} R^{\prime}+2 R^{\prime \prime}+\frac{R^{\prime 2}}{R}\right)\right]                            \\
	G_{t r}           & =-\frac{2}{R}\left(\dot{R}^{\prime}+\dot{\Lambda} R^{\prime}\right)                                                                                                                              \\
	G_{r r}           & =\frac{1}{R^2}\left[R^{\prime 2}-e^{-2 \Lambda}\left(\dot{R}^2+2 R \ddot{R}+1\right)\right]                                                                                                      \\
	G_{\theta \theta} & =\sin ^{-2} \theta G_{\varphi \varphi}=R\left(\dot{R} \dot{\Lambda}+\Lambda^{\prime} R^{\prime} e^{2 \Lambda}+R^{\prime \prime} e^{2 \Lambda}-\ddot{R}+\ddot{\Lambda} R-R \dot{\Lambda}^2\right)
	.\end{align}
We want to consider an isotropic and inhomogeneous barotropic fluid. Thus our energy-momentum tensor will have the form of a perfect fluid
\begin{align}
	\label{perfectfr}
	T_{\mu\nu} & = (\rho(r,t)+ p(r,t) )u_\mu u_\nu + p(r,t) g _{\mu\nu}
\end{align}
where $u_{\mu} u^\mu = -1$ is the velocity vector of the fluid, which we assume to be orthogonal to the spatial hypersurfaces. It is worth mentioning that in Eq.~\eqref{perfectfr}, both the pressure and the energy density depend on the spatial radius since we want to consider a physical metric modeled as a perturbation around the background quantities. Also, the equation of state parameter of the fluid is
\begin{align}
	\label{eqst}
	w & = \frac{p}{\rho}
\end{align}
and we are considering $w \ll 1$ for cold dark matter. In this context, the energy-momentum tensor components are
\begin{align}
	\label{emprojec}
	T_{u u}     & =\rho(t, r)                                         \\
	T^{r}{}_r   & =T^{\theta}{}_\theta=T^{\varphi}{}_\varphi =p(t, r) \\
	T^{r}{}_{u} & = 0,                                                \\
	T_{r u}     & = 0
	,\end{align}
where the indexes $u$ and $r$ indicate both the fluid's velocity direction and the radial direction respectively. Let us now move to the Einstein's equations (EE).

We can redefine our variables as
\begin{align}
	\label{kmdef}
	k(t, r)=1-R^{\prime 2} e^{2 \Lambda}, \quad m(t, r)=\frac{R}{2}\left(\dot{R}^2+k\right)
	,\end{align}
such that the EE will be given by
\begin{align}
	 & k^{\prime} = \kappa RR^{\prime}\left(T_{uu}+T^{r}{}_{r}\right)+2R^{\prime}\left(\ddot{R}+\dot{\Lambda}\dot{R}\right), \\
	 & \dot{k} = \kappa RR^{\prime}T^{r}{}_{u},                                                                              \\
	 & m^{\prime} =\frac{\kappa}{2} R^{2}R^{\prime}T_{uu}-\frac{\kappa}{2} R^{2}\dot{R}T_{ru},                               \\
	 & \dot{m} =\frac{\kappa}{2} R^{2}R^{\prime}T^{r}{}_{u}-\frac{\kappa}{2}\dot{R}R^{2}T^{r}{}_{r}
	.\end{align}
Additionally, the energy conservation of $T^{\mu\nu}$ gives the constraint
\begin{align}
	\label{eqenergy}
	 & \nabla_{\mu}T^{\mu\nu}=0,~\text{or}\nonumber                                                           \\
	 & \nabla_\mu(\rho u^\mu u^\nu+p\mathrm{~h}^{\mu\nu})=\left(\dot{\rho}+3H(\rho+p)\right)u^\nu=0
	,\end{align}
where $\mathrm{~h}^{\mu\nu} = g^{\mu\nu} + u^\mu u^\nu $.
Thus, our system is given by
\begin{align}
	\label{metricgen}
	\mathrm{d} s^2=-\mathrm{d} t^2+\frac{(R^\prime)^{2}}{1 - k} \mathrm{d} r^2+R^2(t, r)\left(\mathrm{d} \theta^2+\sin ^2 \theta \mathrm{d} \varphi^2\right)
\end{align}
with
\begin{align}
	\label{kprime}
	k^{\prime} & =\kappa RR^{\prime}\left(\rho+p\right)+2R^{\prime}\left(\ddot{R}+\dot{\Lambda}\dot{R}\right), \\
	\label{kdot}
	\dot{k}    & = 0,                                                                                          \\
	m^{\prime} & =\frac{\kappa}{2} R^{2}R^{\prime}\rho                                                         \\
	\label{mdot}
	\dot{m}    & =-\frac{\kappa}{2}\dot{R}R^{2}p = -\frac{\kappa}{2}\dot{R}R^{2} w\rho = 0.
\end{align}
plus Eqs.~\eqref{eqst} and~\eqref{eqenergy}. One can replace Eqs.~\eqref{mdot} and
\eqref{kdot} into \eqref{kprime} using the definitions in \eqref{kmdef} yielding the
final equations
\begin{align}
	\label{mprime}
	m^{\prime}  & =\frac{\kappa}{2} R^{2}R^{\prime}\rho \\
	\label{rdot}
	(\dot{R})^2 & = \frac{2m}{R}-k
	.\end{align}

We can recognize in Eqs.~\eqref{mprime} and \eqref{rdot} the Lemaitre-Tolman-Bondi (LTB) class of solutions~\cite{Lemaitre1933}. One may check that the metric in Eq.~\eqref{metricgen} describes the Schwarshild metric if $m$ is constant, the Einstein-de Sitter universe if $R = a(t)r$ and $k=0$ and the closed Friedmann universe if $R= a(t)r$ and $k = r^2$. We want to find analytical solutions for the system without considering a specific type of these local functions. Luckily, the LTB class of solutions is one of the few cases where there is an analytical parametric solution to Einstein's field equations, given by
\begin{align}
	\label{ltb1}
	R(\theta, r) & = \frac{2m}{k}\sin^2\left(\frac{\theta}{2}\right)                          \\
	\label{ltb2}
	t(\theta, r) & = t_1(r) + \frac{m}{k^{\frac{3}{2}}}\left(\theta - \pi - \sin\theta\right)
	.\end{align}
In the above, $\theta$ is a parameter in the range $[-2\pi, 0]$ and $t_1$ is an integration constant\footnote{Our definition of the parameter $\theta$ differs from Ref.~\cite{Martin2020} by $-\pi$ for simplification purposes.}. We still have to find $m$ and $k$, which will require initial conditions. Let us first analyze $m$.

From the right side of Eq.~\eqref{mprime}, we see that $m$ is related to the mass density of a spherical volume. To analyze this quantity, we start by rewriting the energy density from the spherical collapse metric as
\begin{align}
	\rho(t, r) & = \bar{\rho}(t)(1 + \delta(t,r))
	,\end{align}
such that $\delta$ is the density contrast $\delta = \frac{\rho(t,r) - \bar{\rho}(t)}{\bar{\rho}(t)}$ and $\bar{\rho}$ the background energy density.  We consider that there is a small overdensity in our spherical system when compared to the background density. We want to study a top-hat spherical collapse so that outside the over-dense spherical shell characterized by a critical radius $r_c$, the energy density is approximately the background density, i.e., ($r > r_c$)$\rightarrow$ $\rho \approx \bar{\rho}$. We can add this information to the density contrast by making the redefinition
\begin{align}
	\delta(t, r) & \rightarrow \delta(t, r) \Theta(r-r_c)
\end{align}
where $\Theta$ is the Heaviside function. In this context, Eq.~\eqref{mprime} becomes
\begin{align}
	m(r) & = \int_0^r dr_1~ 4\pi GR^2R^\prime \left(\bar{\rho} + \delta \Theta(r-r_c)\right)
	.\end{align}

We still have freedom in our metric to define the local function $R$, which can be fixed
with initial conditions. The simplest choice would be that at initial collapse time
$t_{ini}$, $R(t_{ini},r) = r$, which implies that the 3-dimensional spherical shell is
initially at rest\footnote{Note that any other initial condition is valid and sufficient
	to completely define $m$.}. This leads to
\begin{align}
	\label{mr1}
	m(r) & = \frac{4\pi r^3G \bar{\rho}(t_{ini})}{3}\left(1 + \frac{3}{r^3}\int_0^r dr_1 (r_1)^2\delta(t_{ini}, r_1)\Theta(r-r_c)\right)
	.\end{align}
The integral represents the expected value of the density contrast inside a spherical region with radius $r$ and $m$ gives the total mass inside this region. To evaluate it, we need to assume that at the initial time, our spherical metric possesses the same symmetries as the background and thus it is homogeneous. This means that the density contrast has a uniform distribution and does not depend on position at $t_{ini}$\footnote{This is only true at $t_{ini}$. At later times, the perturbation evolves and depends on the radial position.}. Consequently,
\begin{align}
	\label{deltaini}
	\delta(t_{ini}, r) & = \delta(t_{ini}) \equiv  \delta_{ini}.
\end{align}
We are representing any variable $V$ at initial time $t_{ini}$ as $V(t_{ini})\equiv V_{ini}$ to simplify the notation. Plugging Eq.~\eqref{deltaini} in \eqref{mr1} leads to
\begin{align}
	\label{pbhmassmetric}
	m (r) & =  \left\{\begin{array}{ll}
		                  M\left(\frac{r^3}{r_c^3}\right)                                       & r \leq r_c \\
		                  M+\frac{M}{1+\delta_{ini}}\left(\frac{r^3}{r_{\mathrm{c}}^3}-1\right) & r > r_c
	                  \end{array}\right.,
\end{align}
such that we rewrote the terms as a function of
\begin{align}
	M & =m(r_c)=\frac{4\pi G \bar{\rho}_{ini}r^3_c}{3}\left(1 + \delta_{ini}\right)
	.\end{align}

We may now compute $k$ using the second relation in Eq.~\eqref{kmdef}, which requires us to evaluate $\dot{R}(t, r)$. To do so, we define the inhomogenous Hubble function
\begin{align}
	\label{hdefltb}
	H & \equiv \frac{\dot{R}(t, r)}{R(t, r)}
	.\end{align}
Since we have the vanishing relations in Eqs.~\eqref{mdot} and \eqref{kdot}, $R$ is the
only local degree of freedom that evolves in time and provides the dynamics of the
universe according to Eq.~\eqref{ltb1}. Hence it is natural that we use this quantity to
define the Hubble function.  Assuming that the local metric is homogeneous at $t_{ini}$,
\begin{align}
	\label{hinicond}
	H^2(t_{ini}, r) & = \bar{H}^2(t_{ini}) = \frac{ \kappa \bar{\rho}}{3} = \frac{2 M}{1 +  \delta_{ini}} \frac{1}{r_c^3}
	.\end{align}
Thus, we can use this relation to compute $k$ for $t = t_{ini}$ since this variable does not evolve in time, which leads to
\begin{align}
	k(r) & = \frac{2m}{r} - r^2H^2_{ini}= \left\{\begin{array}{ll}
		                                             2M\frac{\delta_{ini}}{1+\delta_{ini}}\frac{r^2}{r_\mathrm{c}^3} & r \leq r_c \\
		                                             2M\frac{\delta_\mathrm{ini}}{1+\delta_{ini}}\frac{1}{r}         & r > r_c
	                                             \end{array}\right.
	.\end{align}

Now that we have properly computed our local quantities, we can rewrite our solutions in Eq.~\eqref{ltb1} and~\eqref{ltb2} for $r\leq r_c$ as
\begin{align}
	\label{ltb3}
	R(\theta, r)        & = \frac{r (1+\delta_{ini})}{\delta_{ini}}\sin^2\left(\frac{\theta}{2}\right),                             \\
	\label{ltb4}
	\Delta t(\theta,r ) & = \frac{1 + \delta_{ini}}{2H_{ini}\delta^{\frac{3}{2}}(t_{ini})}\left(\theta - \pi - \sin \theta \right),
\end{align}
such that $\Delta t = t_i - t_j$ for arbitrary times and $\theta_{ini}$ is obtained from Eq.~\eqref{ltb1} at $t_{ini}$, i.e,
\begin{align}
	\label{sintheta}
	\sin^2\left(\frac{\theta_{ini}}{2}\right) & = \frac{\delta_{ini}}{1 + \delta_{ini}}
	.\end{align}
We are now able to use these solutions to compute the critical value for the density contrast.
%%%%%%%%%%%%%%%%%%%%%%%%%%%%%%%%%%%%%%%%%%%%%%%%%%%%%%%%%%
\section{Fluid's Gauge}
\label{appgauge}
In the last appendix, we computed the solution for the Einstein equations for an inhomogeneous dust-perfect fluid with a spherically symmetric local metric. To use this solution and compare it with our perturbations defined in Sec.~\ref{linearpert}, we must ensure that measurements in the local metric are in the same Gauge as our perturbations. Since we projected the energy-momentum tensor in the direction of the velocity of the fluid in Eq.~\eqref{emprojec}, we consider the Gauge where the fluid is at rest and we shall now see how to define our perturbation theory in this Gauge choice.

Essentially, following \cite{Vitenti2014covariant}, we want to make sure that the local tensor in \eqref{perfectfr} can be seen as the perturbed energy-momentum tensor $\delta T_{\mu\nu}$. Explicitly, we want
\begin{align}
	T_{\mu \nu}\propto \delta T_{\mu \nu}=(\delta \rho-2 \phi) n_\mu n_\nu+2(\rho+p) n_{(\mu} \Bar{D}_{\nu)}\mathcal{V}+ \left(\delta p \right)\mathrm{~h}_{\mu \nu}+2 p\left(n_{(\mu} \Bar{D}_{\nu)}\mathcal{B}+\Bar{D}_\mu \Bar{D}_\nu \mathcal{E}\right)
	,\end{align}
where we consider no anisotropic pressure. Thus,
the fluid's Gauge is obtained by setting
\begin{align}
	\label{psigauge}
	\mathcal{V}             & = \mathcal{E}= 0 ~~\text{and} \\
	\mathcal{B}\rvert_{t_1} & = 0
	.\end{align}
With these Gauge choices,
\begin{align}
	\label{tensorconect}
	\delta T_{\mu \nu}=(\delta \rho-2 \phi) n_\mu n_\nu+ \left(\delta p \right)\mathrm{~h}_{\mu \nu}+2 p\left(\Bar{D}_\mu \Bar{D}_\nu \mathcal{E}\right)
	.\end{align}

The first choice for $\mathcal{V}$ in Eq.~\eqref{psigauge} assures that the perturbed fluid is at rest with the background universe. The other two conditions guarantee that, at least for an initial time $t_1$, there are no off-diagonal terms and we have an isotropic perturbed fluid. For different times, it is not possible to set a Gauge such that $\mathcal{E}=0$ for $t \neq t_1$~\cite{vitenti2012large}. Nonetheless, we can argue that this term is proportional to the pressure of the fluid which has an almost vanishing value for cold dark matter and thus can be discarded. Also, since our main goal is to make proper measurements of the density contrast that depends on Eq.~\eqref{deltarhoinvariant}, this term is not relevant to our computations. Finally, in the fluid's Gauge, the gauge-invariant density contrast may be interpreted as its physical equivalent, that is,
\begin{align}
	\label{physicalrho}
	\Tilde{\delta \rho} & = \delta \rho + \mathcal{V}\dot{\Bar{\rho}} = \delta \rho\end{align}
where we used Eq.~\eqref{deltarhoinvariant}.
%%%%%%%%%%%%%%%%%%%%%%%%%%%%%%%%%%%%%%%%%%%%%%%%%%%%%%
\acknowledgments

SDPV acknowledges the support of CNPq of Brazil under grant PQ-II 316734/2021-7. EJB acknowledges the support of CAPES under the grant DS 88887.510837/2020-00. LFD acknowledges the support of CAPES under the grant DS 88887.902808/2023-00. We thank Nelson Pinto-Neto and Sheng-Feng Yan for their valuable discussions.


\bibliographystyle{unsrt}
\bibliography{g}

\end{document}
