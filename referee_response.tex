\documentclass[a4paper,11pt]{article}
\pdfoutput=1 % if your are submitting a pdflatex (i.e. if you have
             % images in pdf, png or jpg format)

\usepackage[T1]{fontenc} % if needed

%%%%%%%%%%%%%%%%%%%%%%%%%%%%%% My Packages %%%%%%%%%%%%%%%%%
\usepackage[english]{babel}
\usepackage{braket}
\usepackage{amsmath}
\usepackage{amsfonts}
\usepackage{amssymb}
\usepackage{physics,slashed}
\usepackage{graphicx}
\usepackage[toc,page]{appendix}
\usepackage{hyperref}
\usepackage[noabbrev]{cleveref}

\usepackage{xcolor}
\usepackage{mdframed}

\usepackage[hmarginratio=1:1]{geometry}

\begin{document}

We appreciate the referee’s careful review of our manuscript and the insightful feedback
provided. Below, we provide detailed responses to each point and explain the
corresponding changes made to the manuscript where applicable:

\begin{enumerate}
    \item
          \begin{enumerate}
              \item  To obtain the matter Lagrangian for this model, we use the Schutz
                    formalism, following Ref.~\cite{fluidgeral}. In this framework, the
                    Lagrangian corresponds to the fluid pressure, which depends on the
                    specific entropy and enthalpy. A detailed discussion of this
                    approach is provided in the cited reference. For clarity, we rewrote
                    the description of the Schutz formalism in Section 2.2 and ensured
                    that the references are correctly cited.

              \item Indeed, if the sound speed were to approach zero, we would encounter
                    a divergence problem. This is why we limit ourselves to small but
                    non-zero equations of state for dust, which is in agreement with the
                    literature. Regarding the $\zeta$ field, there are always field
                    configurations that would lead to non-perturbative behavior.
                    However, in the present context, given the initial conditions in
                    equation 3.32, we are considering the adiabatic vacuum for which
                    $\zeta \propto 1/\sqrt{z^2c_s} \propto \sqrt{c_s}$. Therefore, terms
                    of the form $z^2c_s^2\zeta^2$ are $\propto c_s$ while $\zeta^3
                        \propto \sqrt{c_s}^3$ This means that the perturbative expansion is valid
                    for small sound speeds, as discussed in
                    Ref.~\cite{vitenti2012large}.

                    We have reorganized Sections 4.3 and 4.4 and considered a very
                    small, but non-zero, equation of state for dark matter to analyze
                    the pressureless case.
          \end{enumerate}

    \item

          \begin{enumerate}
              \item We improved Sec.~2.2 by adding a more detailed explanation of the
                    Schutz formalism and the fluid Lagrangian. We also included a
                    discussion of the lapse function and how it can be defined as a
                    posteriori for simplicity.

              \item In the improved Sec.~2.2, we have added a brief discussion of the
                    meaning of the effective energy density and pressure in the context
                    of the quantum corrections. In Ref.~\cite{fluidgeral}, the authors
                    show that the energy density and pressure appearing in the perturbed
                    equations are only the classical ones computed with the Bohmian
                    trajectory. Therefore, the equation-of-state parameter is always
                    $w=p_d/\rho_d$.

              \item In the revised Sec.~2.2, we have included the effective equation
                    corresponding to the second Friedmann equation, now presented as Eq. 2.17:
                    $-2\dot{\bar{H}} = \bar{\rho}(1+w) - 2\rho_q.$
                    Notably, the right-hand side of this equation approaches zero around the
                    bounce. However, as discussed previously in Ref.~\cite{fluidgeral} it
                    has been shown that the energy density and pressure in the second-order
                    Lagrangian for the perturbations are only the classical components
                    evaluated along the Bohmian trajectory. Consequently, this term is never
                    zero, ensuring that the model is free from ghost instability issues.
              \item In Ref.~33 the author chooses a particular form for the wave
                    function at the bounce. This is indeed not an initial condition, but
                    a choice made to simplify the analysis. The same wave-function
                    propagated to $T\to-\infty$ would have an extra phase factor
                    proportional to the $q^2$, exactly like a wave function of a
                    free Gaussian package in 1d. One can always start with a different
                    wave function and define the bounce as the time where the phase
                    factor proportional to $q^2$ is zero. Again, this is equivalent to a
                    free Gaussian package in 1d. Thus, in short, it has to start at a large
                    but finite negative time, with a Gaussian wave-function containing a
                    phase factor proportional to $q^2$. Then, the bounce is defined as
                    the time when this phase factor is zero. We have included a brief
                    discussion of this point in the revised manuscript.
              \item In our approach, we use a perturbative method: the background is
                    solved first, and the Bohmian trajectory is derived from the
                    background wave function. Perturbations are then computed
                    conditioned on this Bohmian trajectory. In practice, during the
                    contraction phase, the significant part of the power spectrum is
                    generated in the classical regime, effectively following a
                    semi-classical approach. Quantum effects on the background become
                    relevant only near the bounce, where the relevant modes are on
                    super-Hubble scales. If the background is in a superposition of
                    Gaussian states, the perturbations would naturally reflect this
                    superposition. However, this aspect lies outside the scope of our
                    work, as we are focused here on black hole formation during the
                    contraction phase.

          \end{enumerate}

    \item In general, when both the background and the perturbations are quantized it is necessary
          to develop the second order Lagrangian for the perturbations without assuming the background
          is classical. Once this is done, one can show that for particular choices of the wave function
          for the whole system (background plus perturbations), the perturbations can be treated as
          quantum fields evolving around the Bohmian trajectory. We included a brief discussion of this
          point in the revised manuscript, Sec.~3.3.

    \item In Ref.~\cite{vitenti2012large}, it is shown that for a Friedmann background with
          scalar perturbations, an appropriate gauge choice ensures the perturbative
          series remains valid in the bounce model. Nevertheless, this only shows that
          vacuum initial conditions around an initial Friedmann background are
          consistent with the perturbative expansion. The referee is correct in pointing
          out that the perturbative expansion may not be valid for more general initial
          conditions. In our study, we focus on this simpler model, leaving the
          exploration of a homogeneous and anisotropic background, specifically
          Bianchi-I models, for future work. A brief discussion of this possibility has
          been included in the conclusion.

    \item To simplify the presentation, we set the speed of light $c=1$ in the main
          text. We have now included a brief discussion of this choice in the revised
          manuscript. We also clarified the definition of energy density and perturbation
          in the main text.
\end{enumerate}

\bibliographystyle{unsrt}
\bibliography{g}
\end{document}
